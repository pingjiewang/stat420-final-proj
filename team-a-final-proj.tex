\documentclass[]{article}
\usepackage{lmodern}
\usepackage{amssymb,amsmath}
\usepackage{ifxetex,ifluatex}
\usepackage{fixltx2e} % provides \textsubscript
\ifnum 0\ifxetex 1\fi\ifluatex 1\fi=0 % if pdftex
  \usepackage[T1]{fontenc}
  \usepackage[utf8]{inputenc}
\else % if luatex or xelatex
  \ifxetex
    \usepackage{mathspec}
  \else
    \usepackage{fontspec}
  \fi
  \defaultfontfeatures{Ligatures=TeX,Scale=MatchLowercase}
\fi
% use upquote if available, for straight quotes in verbatim environments
\IfFileExists{upquote.sty}{\usepackage{upquote}}{}
% use microtype if available
\IfFileExists{microtype.sty}{%
\usepackage{microtype}
\UseMicrotypeSet[protrusion]{basicmath} % disable protrusion for tt fonts
}{}
\usepackage[margin=1in]{geometry}
\usepackage{hyperref}
\PassOptionsToPackage{usenames,dvipsnames}{color} % color is loaded by hyperref
\hypersetup{unicode=true,
            pdftitle={Building an Used Car Price Prediction Model for Germany eBay listings},
            pdfauthor={Richa Gupta/Sandro Tanis/Ping Wang},
            colorlinks=true,
            linkcolor=Maroon,
            citecolor=Blue,
            urlcolor=cyan,
            breaklinks=true}
\urlstyle{same}  % don't use monospace font for urls
\usepackage{color}
\usepackage{fancyvrb}
\newcommand{\VerbBar}{|}
\newcommand{\VERB}{\Verb[commandchars=\\\{\}]}
\DefineVerbatimEnvironment{Highlighting}{Verbatim}{commandchars=\\\{\}}
% Add ',fontsize=\small' for more characters per line
\usepackage{framed}
\definecolor{shadecolor}{RGB}{248,248,248}
\newenvironment{Shaded}{\begin{snugshade}}{\end{snugshade}}
\newcommand{\AlertTok}[1]{\textcolor[rgb]{0.94,0.16,0.16}{#1}}
\newcommand{\AnnotationTok}[1]{\textcolor[rgb]{0.56,0.35,0.01}{\textbf{\textit{#1}}}}
\newcommand{\AttributeTok}[1]{\textcolor[rgb]{0.77,0.63,0.00}{#1}}
\newcommand{\BaseNTok}[1]{\textcolor[rgb]{0.00,0.00,0.81}{#1}}
\newcommand{\BuiltInTok}[1]{#1}
\newcommand{\CharTok}[1]{\textcolor[rgb]{0.31,0.60,0.02}{#1}}
\newcommand{\CommentTok}[1]{\textcolor[rgb]{0.56,0.35,0.01}{\textit{#1}}}
\newcommand{\CommentVarTok}[1]{\textcolor[rgb]{0.56,0.35,0.01}{\textbf{\textit{#1}}}}
\newcommand{\ConstantTok}[1]{\textcolor[rgb]{0.00,0.00,0.00}{#1}}
\newcommand{\ControlFlowTok}[1]{\textcolor[rgb]{0.13,0.29,0.53}{\textbf{#1}}}
\newcommand{\DataTypeTok}[1]{\textcolor[rgb]{0.13,0.29,0.53}{#1}}
\newcommand{\DecValTok}[1]{\textcolor[rgb]{0.00,0.00,0.81}{#1}}
\newcommand{\DocumentationTok}[1]{\textcolor[rgb]{0.56,0.35,0.01}{\textbf{\textit{#1}}}}
\newcommand{\ErrorTok}[1]{\textcolor[rgb]{0.64,0.00,0.00}{\textbf{#1}}}
\newcommand{\ExtensionTok}[1]{#1}
\newcommand{\FloatTok}[1]{\textcolor[rgb]{0.00,0.00,0.81}{#1}}
\newcommand{\FunctionTok}[1]{\textcolor[rgb]{0.00,0.00,0.00}{#1}}
\newcommand{\ImportTok}[1]{#1}
\newcommand{\InformationTok}[1]{\textcolor[rgb]{0.56,0.35,0.01}{\textbf{\textit{#1}}}}
\newcommand{\KeywordTok}[1]{\textcolor[rgb]{0.13,0.29,0.53}{\textbf{#1}}}
\newcommand{\NormalTok}[1]{#1}
\newcommand{\OperatorTok}[1]{\textcolor[rgb]{0.81,0.36,0.00}{\textbf{#1}}}
\newcommand{\OtherTok}[1]{\textcolor[rgb]{0.56,0.35,0.01}{#1}}
\newcommand{\PreprocessorTok}[1]{\textcolor[rgb]{0.56,0.35,0.01}{\textit{#1}}}
\newcommand{\RegionMarkerTok}[1]{#1}
\newcommand{\SpecialCharTok}[1]{\textcolor[rgb]{0.00,0.00,0.00}{#1}}
\newcommand{\SpecialStringTok}[1]{\textcolor[rgb]{0.31,0.60,0.02}{#1}}
\newcommand{\StringTok}[1]{\textcolor[rgb]{0.31,0.60,0.02}{#1}}
\newcommand{\VariableTok}[1]{\textcolor[rgb]{0.00,0.00,0.00}{#1}}
\newcommand{\VerbatimStringTok}[1]{\textcolor[rgb]{0.31,0.60,0.02}{#1}}
\newcommand{\WarningTok}[1]{\textcolor[rgb]{0.56,0.35,0.01}{\textbf{\textit{#1}}}}
\usepackage{graphicx,grffile}
\makeatletter
\def\maxwidth{\ifdim\Gin@nat@width>\linewidth\linewidth\else\Gin@nat@width\fi}
\def\maxheight{\ifdim\Gin@nat@height>\textheight\textheight\else\Gin@nat@height\fi}
\makeatother
% Scale images if necessary, so that they will not overflow the page
% margins by default, and it is still possible to overwrite the defaults
% using explicit options in \includegraphics[width, height, ...]{}
\setkeys{Gin}{width=\maxwidth,height=\maxheight,keepaspectratio}
\IfFileExists{parskip.sty}{%
\usepackage{parskip}
}{% else
\setlength{\parindent}{0pt}
\setlength{\parskip}{6pt plus 2pt minus 1pt}
}
\setlength{\emergencystretch}{3em}  % prevent overfull lines
\providecommand{\tightlist}{%
  \setlength{\itemsep}{0pt}\setlength{\parskip}{0pt}}
\setcounter{secnumdepth}{0}
% Redefines (sub)paragraphs to behave more like sections
\ifx\paragraph\undefined\else
\let\oldparagraph\paragraph
\renewcommand{\paragraph}[1]{\oldparagraph{#1}\mbox{}}
\fi
\ifx\subparagraph\undefined\else
\let\oldsubparagraph\subparagraph
\renewcommand{\subparagraph}[1]{\oldsubparagraph{#1}\mbox{}}
\fi

%%% Use protect on footnotes to avoid problems with footnotes in titles
\let\rmarkdownfootnote\footnote%
\def\footnote{\protect\rmarkdownfootnote}

%%% Change title format to be more compact
\usepackage{titling}

% Create subtitle command for use in maketitle
\providecommand{\subtitle}[1]{
  \posttitle{
    \begin{center}\large#1\end{center}
    }
}

\setlength{\droptitle}{-2em}

  \title{Building an Used Car Price Prediction Model for Germany eBay listings}
    \pretitle{\vspace{\droptitle}\centering\huge}
  \posttitle{\par}
    \author{Richa Gupta/Sandro Tanis/Ping Wang}
    \preauthor{\centering\large\emph}
  \postauthor{\par}
      \predate{\centering\large\emph}
  \postdate{\par}
    \date{Aug 07, 2020}


\begin{document}
\maketitle

\begin{center}\rule{0.5\linewidth}{\linethickness}\end{center}

\hypertarget{introduction}{%
\section{1. Introduction}\label{introduction}}

\hypertarget{methods}{%
\section{2. Methods}\label{methods}}

\hypertarget{data-cleanning-and-exploration}{%
\subsection{2.1 Data cleanning and
exploration}\label{data-cleanning-and-exploration}}

Loading for raw data

\begin{Shaded}
\begin{Highlighting}[]
\KeywordTok{library}\NormalTok{(readr)}
\NormalTok{autos_raw <-}\StringTok{ }\KeywordTok{read_csv}\NormalTok{(}\StringTok{"autos.csv"}\NormalTok{)}
\KeywordTok{attr}\NormalTok{(autos_raw, }\StringTok{'spec'}\NormalTok{) <-}\StringTok{ }\OtherTok{NULL}
\KeywordTok{attr}\NormalTok{(autos_raw, }\StringTok{'problems'}\NormalTok{) <-}\StringTok{ }\OtherTok{NULL}

\KeywordTok{str}\NormalTok{(autos_raw)}
\end{Highlighting}
\end{Shaded}

\begin{verbatim}
## Classes 'spec_tbl_df', 'tbl_df', 'tbl' and 'data.frame': 354687 obs. of  20 variables:
##  $ dateCrawled        : POSIXct, format: "2016-03-24 11:52:17" "2016-03-24 10:58:45" ...
##  $ name               : chr  "Golf_3_1.6" "A5_Sportback_2.7_Tdi" "Jeep_Grand_Cherokee_\"Overland\"" "GOLF_4_1_4__3T\xdcRER" ...
##  $ seller             : chr  "privat" "privat" "privat" "privat" ...
##  $ offerType          : chr  "Angebot" "Angebot" "Angebot" "Angebot" ...
##  $ price              : num  480 18300 9800 1500 3600 650 2200 0 14500 999 ...
##  $ abtest             : chr  "test" "test" "test" "test" ...
##  $ vehicleType        : chr  NA "coupe" "suv" "kleinwagen" ...
##  $ yearOfRegistration : num  1993 2011 2004 2001 2008 ...
##  $ gearbox            : chr  "manuell" "manuell" "automatik" "manuell" ...
##  $ powerPS            : num  0 190 163 75 69 102 109 50 125 101 ...
##  $ model              : chr  "golf" NA "grand" "golf" ...
##  $ kilometer          : num  150000 125000 125000 150000 90000 150000 150000 40000 30000 150000 ...
##  $ monthOfRegistration: num  0 5 8 6 7 10 8 7 8 0 ...
##  $ fuelType           : chr  "benzin" "diesel" "diesel" "benzin" ...
##  $ brand              : chr  "volkswagen" "audi" "jeep" "volkswagen" ...
##  $ notRepairedDamage  : chr  NA "ja" NA "nein" ...
##  $ dateCreated        : POSIXct, format: "2016-03-24" "2016-03-24" ...
##  $ nrOfPictures       : num  0 0 0 0 0 0 0 0 0 0 ...
##  $ postalCode         : chr  "70435" "66954" "90480" "91074" ...
##  $ lastSeen           : POSIXct, format: "2016-04-07 03:16:57" "2016-04-07 01:46:50" ...
\end{verbatim}

\hypertarget{data-cleaning-and-preparation}{%
\subsection{2.1.1 Data cleaning and
preparation}\label{data-cleaning-and-preparation}}

\begin{Shaded}
\begin{Highlighting}[]
\NormalTok{autos=}\KeywordTok{subset}\NormalTok{(autos_raw, price}\OperatorTok{>}\DecValTok{0} \OperatorTok{&}\StringTok{ }\NormalTok{yearOfRegistration}\OperatorTok{>=}\DecValTok{2000} \OperatorTok{&}\StringTok{ }\NormalTok{offerType}\OperatorTok{==}\StringTok{"Angebot"} \OperatorTok{&}\StringTok{ }\NormalTok{seller}\OperatorTok{==}\StringTok{"privat"}\NormalTok{)}
\NormalTok{columns_remove=}\KeywordTok{c}\NormalTok{(}\StringTok{"postalCode"}\NormalTok{,}\StringTok{"lastSeen"}\NormalTok{, }\StringTok{"nrOfPictures"}\NormalTok{,}\StringTok{"dateCreated"}\NormalTok{,}\StringTok{"dateCrawled"}\NormalTok{,}\StringTok{"monthOfRegistration"}\NormalTok{,}\StringTok{"offerType"}\NormalTok{,}\StringTok{"seller"}\NormalTok{)   }
\NormalTok{columns_numeric =}\StringTok{ }\KeywordTok{c}\NormalTok{(}\StringTok{"price"}\NormalTok{,}\StringTok{"powerPS"}\NormalTok{, }\StringTok{"kilometer"}\NormalTok{,}\StringTok{"yearOfRegistration"}\NormalTok{)}
\NormalTok{columns_factor =}\StringTok{ }\KeywordTok{c}\NormalTok{(}\StringTok{"abtest"}\NormalTok{,}\StringTok{"vehicleType"}\NormalTok{,}\StringTok{"gearbox"}\NormalTok{,}\StringTok{"model"}\NormalTok{,}\StringTok{"brand"}\NormalTok{,}\StringTok{"fuelType"}\NormalTok{,}\StringTok{"notRepairedDamage"}\NormalTok{ )}
\NormalTok{other_columns =}\StringTok{ }\KeywordTok{c}\NormalTok{(}\StringTok{"name"}\NormalTok{) }

\CommentTok{#autos[,columns_factor]=lapply(autos[,columns_factor], as.factor)}
\NormalTok{autos =}\StringTok{ }\NormalTok{autos[, }\OperatorTok{-}\KeywordTok{which}\NormalTok{(}\KeywordTok{names}\NormalTok{(autos) }\OperatorTok\StringTok{ }\NormalTok{columns_remove)]}
\NormalTok{autos =}\StringTok{ }\NormalTok{autos[, }\OperatorTok{-}\KeywordTok{which}\NormalTok{(}\KeywordTok{names}\NormalTok{(autos) }\OperatorTok\StringTok{ }\NormalTok{other_columns)]}
\KeywordTok{str}\NormalTok{(autos)}
\end{Highlighting}
\end{Shaded}

\begin{verbatim}
## Classes 'tbl_df', 'tbl' and 'data.frame':    250368 obs. of  11 variables:
##  $ price             : num  18300 9800 1500 3600 2200 ...
##  $ abtest            : chr  "test" "test" "test" "test" ...
##  $ vehicleType       : chr  "coupe" "suv" "kleinwagen" "kleinwagen" ...
##  $ yearOfRegistration: num  2011 2004 2001 2008 2004 ...
##  $ gearbox           : chr  "manuell" "automatik" "manuell" "manuell" ...
##  $ powerPS           : num  190 163 75 69 109 125 105 140 131 190 ...
##  $ model             : chr  NA "grand" "golf" "fabia" ...
##  $ kilometer         : num  125000 125000 150000 90000 150000 30000 150000 150000 150000 70000 ...
##  $ fuelType          : chr  "diesel" "diesel" "benzin" "diesel" ...
##  $ brand             : chr  "audi" "jeep" "volkswagen" "skoda" ...
##  $ notRepairedDamage : chr  "ja" NA "nein" "nein" ...
\end{verbatim}

\hypertarget{data-exploration-to-find-the-relationship-between-continous-variables}{%
\subsection{2.1.2 Data exploration to find the relationship between
continous
variables}\label{data-exploration-to-find-the-relationship-between-continous-variables}}

\begin{Shaded}
\begin{Highlighting}[]
\KeywordTok{library}\NormalTok{(dplyr)}
\end{Highlighting}
\end{Shaded}

\begin{verbatim}
## 
## Attaching package: 'dplyr'
\end{verbatim}

\begin{verbatim}
## The following objects are masked from 'package:stats':
## 
##     filter, lag
\end{verbatim}

\begin{verbatim}
## The following objects are masked from 'package:base':
## 
##     intersect, setdiff, setequal, union
\end{verbatim}

\begin{Shaded}
\begin{Highlighting}[]
\NormalTok{autos=}\KeywordTok{na.omit}\NormalTok{(autos)}
\NormalTok{autos_factor_groups=autos }\OperatorTok\StringTok{ }\KeywordTok{count}\NormalTok{ (abtest,vehicleType,gearbox,model,brand,fuelType,notRepairedDamage)}
\NormalTok{autos_factor_groups=autos_factor_groups[}\KeywordTok{order}\NormalTok{(autos_factor_groups}\OperatorTok{$}\NormalTok{n,}\DataTypeTok{decreasing =} \OtherTok{TRUE}\NormalTok{),]}
\NormalTok{group1=autos_factor_groups[}\DecValTok{2}\NormalTok{,]}
\NormalTok{group1.size =}\StringTok{ }\NormalTok{group1}\OperatorTok{$}\NormalTok{n}
\NormalTok{group1=}\KeywordTok{subset}\NormalTok{(group1, }\DataTypeTok{select =} \OperatorTok{-}\KeywordTok{c}\NormalTok{(n) )}

\NormalTok{autos_}\DecValTok{1}\NormalTok{=autos}
\NormalTok{cols=}\KeywordTok{names}\NormalTok{(group1)}
\ControlFlowTok{for}\NormalTok{ (i }\ControlFlowTok{in} \DecValTok{1}\OperatorTok{:}\KeywordTok{ncol}\NormalTok{(group1))\{}
\NormalTok{  idx =}\StringTok{ }\NormalTok{autos_}\DecValTok{1}\NormalTok{[,cols[i]]}\OperatorTok{==}\NormalTok{group1[[i]]}
\NormalTok{  autos_}\DecValTok{1}\NormalTok{=autos_}\DecValTok{1}\NormalTok{[idx,]}
\NormalTok{\}}

\NormalTok{autos_}\DecValTok{1}\NormalTok{=}\KeywordTok{subset}\NormalTok{(autos_}\DecValTok{1}\NormalTok{, }\DataTypeTok{select =}\NormalTok{ columns_numeric )}
  

\CommentTok{#}
\KeywordTok{pairs}\NormalTok{(autos_}\DecValTok{1}\NormalTok{,}\DataTypeTok{col=}\StringTok{"dodgerblue"}\NormalTok{)}
\end{Highlighting}
\end{Shaded}

\includegraphics{team-a-final-proj_files/figure-latex/unnamed-chunk-3-1.pdf}

\hypertarget{autos_1-has-been-isolated-to-better-find-a-model-format-for-continuous-variables}{%
\subsubsection{Autos\_1 has been isolated to better find a model format
for continuous
variables}\label{autos_1-has-been-isolated-to-better-find-a-model-format-for-continuous-variables}}

Please use \textbf{autos\_1} for determining box-cox tranformations,it
only contains continuous variables

\begin{Shaded}
\begin{Highlighting}[]
\KeywordTok{par}\NormalTok{(}\DataTypeTok{mfrow=}\KeywordTok{c}\NormalTok{(}\DecValTok{1}\NormalTok{,}\DecValTok{2}\NormalTok{))}
\NormalTok{model_add =}\StringTok{ }\KeywordTok{lm}\NormalTok{(price }\OperatorTok{~}\StringTok{ }\NormalTok{powerPS }\OperatorTok{+}\StringTok{ }\NormalTok{kilometer}\OperatorTok{+}\NormalTok{yearOfRegistration , }\DataTypeTok{data =}\NormalTok{ autos_}\DecValTok{1}\NormalTok{)}
\CommentTok{#fitted vs residual}
\KeywordTok{plot}\NormalTok{(}\KeywordTok{fitted}\NormalTok{(model_add), }\KeywordTok{resid}\NormalTok{(model_add), }\DataTypeTok{col =} \StringTok{"dodgerblue"}\NormalTok{, }\DataTypeTok{pch =} \DecValTok{20}\NormalTok{,}
     \DataTypeTok{xlab =} \StringTok{"Fitted"}\NormalTok{, }\DataTypeTok{ylab =} \StringTok{"Residuals"}\NormalTok{, }\DataTypeTok{main =} \StringTok{"Fitted versus Residuals of additive"}\NormalTok{)}
\KeywordTok{abline}\NormalTok{(}\DataTypeTok{h =} \DecValTok{0}\NormalTok{, }\DataTypeTok{col =} \StringTok{"darkorange"}\NormalTok{, }\DataTypeTok{lwd =} \DecValTok{2}\NormalTok{)}

\CommentTok{#qqplot}
\KeywordTok{qqnorm}\NormalTok{(}\KeywordTok{resid}\NormalTok{(model_add), }\DataTypeTok{main =} \StringTok{"Normal Q-Q Plot of additive"}\NormalTok{, }\DataTypeTok{col =} \StringTok{"dodgerblue"}\NormalTok{)}
\KeywordTok{qqline}\NormalTok{(}\KeywordTok{resid}\NormalTok{(model_add), }\DataTypeTok{col =} \StringTok{"dodgerblue"}\NormalTok{, }\DataTypeTok{lwd =} \DecValTok{2}\NormalTok{)}
\end{Highlighting}
\end{Shaded}

\includegraphics{team-a-final-proj_files/figure-latex/unnamed-chunk-4-1.pdf}

\hypertarget{perform-box-cox-tranformation-on-continuous-variables-only-to-estabish-transformation-form}{%
\subsection{2.1.3 perform box-cox tranformation on continuous variables
only (to estabish transformation form
)}\label{perform-box-cox-tranformation-on-continuous-variables-only-to-estabish-transformation-form}}

\begin{Shaded}
\begin{Highlighting}[]
\KeywordTok{library}\NormalTok{(MASS)}
\end{Highlighting}
\end{Shaded}

\begin{verbatim}
## 
## Attaching package: 'MASS'
\end{verbatim}

\begin{verbatim}
## The following object is masked from 'package:dplyr':
## 
##     select
\end{verbatim}

\begin{Shaded}
\begin{Highlighting}[]
\KeywordTok{library}\NormalTok{(lmtest)}
\end{Highlighting}
\end{Shaded}

\begin{verbatim}
## Loading required package: zoo
\end{verbatim}

\begin{verbatim}
## 
## Attaching package: 'zoo'
\end{verbatim}

\begin{verbatim}
## The following objects are masked from 'package:base':
## 
##     as.Date, as.Date.numeric
\end{verbatim}

\begin{Shaded}
\begin{Highlighting}[]
\KeywordTok{par}\NormalTok{(}\DataTypeTok{mfrow=}\KeywordTok{c}\NormalTok{(}\DecValTok{1}\NormalTok{,}\DecValTok{1}\NormalTok{))}
\NormalTok{out=}\KeywordTok{boxcox}\NormalTok{(model_add, }\DataTypeTok{plotit =} \OtherTok{TRUE}\NormalTok{, }\DataTypeTok{lambda =} \KeywordTok{seq}\NormalTok{(}\OperatorTok{-}\FloatTok{0.5}\NormalTok{, }\FloatTok{2.0}\NormalTok{, }\DataTypeTok{by =} \FloatTok{0.1}\NormalTok{))}
\end{Highlighting}
\end{Shaded}

\includegraphics{team-a-final-proj_files/figure-latex/unnamed-chunk-5-1.pdf}

\begin{Shaded}
\begin{Highlighting}[]
\NormalTok{( }\DataTypeTok{lambda=}\NormalTok{out}\OperatorTok{$}\NormalTok{x[}\KeywordTok{which.max}\NormalTok{(out}\OperatorTok{$}\NormalTok{y)] )}
\end{Highlighting}
\end{Shaded}

\begin{verbatim}
## [1] 0.3081
\end{verbatim}

\begin{Shaded}
\begin{Highlighting}[]
\NormalTok{model_cox_start=}\KeywordTok{lm}\NormalTok{( (price}\OperatorTok{^}\NormalTok{lambda}\DecValTok{-1}\NormalTok{)}\OperatorTok{/}\NormalTok{lambda }\OperatorTok{~}\StringTok{ }\NormalTok{powerPS }\OperatorTok{+}\StringTok{ }\NormalTok{kilometer}\OperatorTok{+}\NormalTok{yearOfRegistration}\OperatorTok{+}\KeywordTok{I}\NormalTok{(powerPS}\OperatorTok{^}\DecValTok{2}\NormalTok{) }\OperatorTok{+}\StringTok{ }\KeywordTok{I}\NormalTok{(kilometer}\OperatorTok{^}\DecValTok{2}\NormalTok{) }\OperatorTok{+}\StringTok{ }\KeywordTok{I}\NormalTok{(}\KeywordTok{log}\NormalTok{(kilometer)), }\DataTypeTok{data =}\NormalTok{ autos_}\DecValTok{1}\NormalTok{)}
\NormalTok{model_cox =}\StringTok{ }\KeywordTok{step}\NormalTok{(model_cox_start,}\DataTypeTok{trace=}\DecValTok{0}\NormalTok{)}
\CommentTok{#model_cox}
\CommentTok{#model_cox2 = step(model_cox_start,k=log(nrow(autos_1)),trace=0)}
\CommentTok{#model_cox2}

\CommentTok{#model_add_cox=lm((price^lambda-1)/lambda ~ powerPS, data = autos_1)}

\KeywordTok{summary}\NormalTok{(model_cox)}
\end{Highlighting}
\end{Shaded}

\begin{verbatim}
## 
## Call:
## lm(formula = (price^lambda - 1)/lambda ~ powerPS + yearOfRegistration + 
##     I(powerPS^2) + I(kilometer^2) + I(log(kilometer)), data = autos_1)
## 
## Residuals:
##     Min      1Q  Median      3Q     Max 
## -30.408  -1.712   0.036   1.653  16.253 
## 
## Coefficients:
##                     Estimate Std. Error t value Pr(>|t|)    
## (Intercept)        -3.43e+03   5.32e+01  -64.52   <2e-16 ***
## powerPS             7.42e-02   1.38e-03   53.65   <2e-16 ***
## yearOfRegistration  1.73e+00   2.63e-02   65.68   <2e-16 ***
## I(powerPS^2)       -5.79e-06   1.20e-07  -48.13   <2e-16 ***
## I(kilometer^2)     -2.67e-10   1.80e-11  -14.83   <2e-16 ***
## I(log(kilometer))   3.99e-01   2.25e-01    1.78    0.076 .  
## ---
## Signif. codes:  0 '***' 0.001 '**' 0.01 '*' 0.05 '.' 0.1 ' ' 1
## 
## Residual standard error: 3.18 on 2023 degrees of freedom
## Multiple R-squared:  0.918,  Adjusted R-squared:  0.918 
## F-statistic: 4.52e+03 on 5 and 2023 DF,  p-value: <2e-16
\end{verbatim}

\begin{Shaded}
\begin{Highlighting}[]
\KeywordTok{par}\NormalTok{(}\DataTypeTok{mfrow=}\KeywordTok{c}\NormalTok{(}\DecValTok{1}\NormalTok{,}\DecValTok{2}\NormalTok{))}
\KeywordTok{plot}\NormalTok{(}\KeywordTok{fitted}\NormalTok{(model_cox), }\KeywordTok{resid}\NormalTok{(model_cox), }\DataTypeTok{col =} \StringTok{"dodgerblue"}\NormalTok{, }
     \DataTypeTok{xlab =} \StringTok{"Fitted"}\NormalTok{, }\DataTypeTok{ylab =} \StringTok{"Residuals"}\NormalTok{, }\DataTypeTok{main =} \StringTok{"Fitted versus Residuals with Box-cox"}\NormalTok{)}
\KeywordTok{abline}\NormalTok{(}\DataTypeTok{h =} \DecValTok{0}\NormalTok{, }\DataTypeTok{col =} \StringTok{"darkorange"}\NormalTok{, }\DataTypeTok{lwd =} \DecValTok{2}\NormalTok{)}
\KeywordTok{qqnorm}\NormalTok{(}\KeywordTok{resid}\NormalTok{(model_cox), }\DataTypeTok{main =} \StringTok{"Normal Q-Q Plot with Box-cox"}\NormalTok{, }\DataTypeTok{col =} \StringTok{"dodgerblue"}\NormalTok{)}
\KeywordTok{qqline}\NormalTok{(}\KeywordTok{resid}\NormalTok{(model_cox), }\DataTypeTok{col =} \StringTok{"dodgerblue"}\NormalTok{, }\DataTypeTok{lwd =} \DecValTok{2}\NormalTok{)}
\end{Highlighting}
\end{Shaded}

\includegraphics{team-a-final-proj_files/figure-latex/unnamed-chunk-5-2.pdf}

\begin{Shaded}
\begin{Highlighting}[]
\KeywordTok{shapiro.test}\NormalTok{(}\KeywordTok{resid}\NormalTok{(model_cox))}
\end{Highlighting}
\end{Shaded}

\begin{verbatim}
## 
##  Shapiro-Wilk normality test
## 
## data:  resid(model_cox)
## W = 0.94, p-value <2e-16
\end{verbatim}

\begin{Shaded}
\begin{Highlighting}[]
\KeywordTok{bptest}\NormalTok{(model_cox)}
\end{Highlighting}
\end{Shaded}

\begin{verbatim}
## 
##  studentized Breusch-Pagan test
## 
## data:  model_cox
## BP = 78, df = 5, p-value = 2e-15
\end{verbatim}

\hypertarget{tempory-block---refit-the-multiple-regression-model-without-any-influential-points}{%
\subsubsection{tempory block - refit the multiple regression model
without any influential
points}\label{tempory-block---refit-the-multiple-regression-model-without-any-influential-points}}

\begin{Shaded}
\begin{Highlighting}[]
\CommentTok{#finding influenctial}
\NormalTok{mod_cook =}\StringTok{ }\KeywordTok{cooks.distance}\NormalTok{(model_cox)}
\NormalTok{n=}\KeywordTok{length}\NormalTok{(}\KeywordTok{resid}\NormalTok{(model_cox))}
\NormalTok{high_infl =}\StringTok{ }\NormalTok{mod_cook }\OperatorTok{>}\StringTok{ }\DecValTok{4} \OperatorTok{/}\StringTok{ }\NormalTok{n}
\KeywordTok{sum}\NormalTok{(high_infl)     }
\end{Highlighting}
\end{Shaded}

\begin{verbatim}
## [1] 96
\end{verbatim}

\begin{Shaded}
\begin{Highlighting}[]
\KeywordTok{mean}\NormalTok{(high_infl)}
\end{Highlighting}
\end{Shaded}

\begin{verbatim}
## [1] 0.04731
\end{verbatim}

\begin{Shaded}
\begin{Highlighting}[]
\CommentTok{#Refit the multiple regression model without any influential points}
\NormalTok{formula=}\KeywordTok{as.formula}\NormalTok{(}\KeywordTok{as.character}\NormalTok{(model_cox}\OperatorTok{$}\NormalTok{call[}\DecValTok{2}\NormalTok{]))}
\NormalTok{model_cox_sub =}\StringTok{ }\KeywordTok{lm}\NormalTok{(formula, }\DataTypeTok{data =}\NormalTok{ autos_}\DecValTok{1}\NormalTok{, }\DataTypeTok{subset =} \OperatorTok{!}\NormalTok{high_infl)}
\KeywordTok{par}\NormalTok{(}\DataTypeTok{mfrow=}\KeywordTok{c}\NormalTok{(}\DecValTok{1}\NormalTok{,}\DecValTok{2}\NormalTok{))}
\KeywordTok{plot}\NormalTok{(}\KeywordTok{fitted}\NormalTok{(model_cox_sub), }\KeywordTok{resid}\NormalTok{(model_cox_sub), }\DataTypeTok{col =} \StringTok{"dodgerblue"}\NormalTok{, }
     \DataTypeTok{xlab =} \StringTok{"Fitted"}\NormalTok{, }\DataTypeTok{ylab =} \StringTok{"Residuals"}\NormalTok{, }\DataTypeTok{main =} \StringTok{"Fitted versus Residuals with Box-cox"}\NormalTok{)}
\KeywordTok{abline}\NormalTok{(}\DataTypeTok{h =} \DecValTok{0}\NormalTok{, }\DataTypeTok{col =} \StringTok{"darkorange"}\NormalTok{, }\DataTypeTok{lwd =} \DecValTok{2}\NormalTok{)}
\KeywordTok{qqnorm}\NormalTok{(}\KeywordTok{resid}\NormalTok{(model_cox_sub), }\DataTypeTok{main =} \StringTok{"Normal Q-Q Plot with Box-cox"}\NormalTok{, }\DataTypeTok{col =} \StringTok{"dodgerblue"}\NormalTok{)}
\KeywordTok{qqline}\NormalTok{(}\KeywordTok{resid}\NormalTok{(model_cox_sub), }\DataTypeTok{col =} \StringTok{"dodgerblue"}\NormalTok{, }\DataTypeTok{lwd =} \DecValTok{2}\NormalTok{)}
\end{Highlighting}
\end{Shaded}

\includegraphics{team-a-final-proj_files/figure-latex/unnamed-chunk-6-1.pdf}

\begin{Shaded}
\begin{Highlighting}[]
\KeywordTok{summary}\NormalTok{(model_cox_sub)}
\end{Highlighting}
\end{Shaded}

\begin{verbatim}
## 
## Call:
## lm(formula = formula, data = autos_1, subset = !high_infl)
## 
## Residuals:
##    Min     1Q Median     3Q    Max 
## -8.774 -1.536  0.074  1.623  9.027 
## 
## Coefficients:
##                     Estimate Std. Error t value Pr(>|t|)    
## (Intercept)        -3.39e+03   4.60e+01  -73.85   <2e-16 ***
## powerPS             6.31e-02   4.25e-03   14.85   <2e-16 ***
## yearOfRegistration  1.71e+00   2.26e-02   75.54   <2e-16 ***
## I(powerPS^2)        3.47e-05   1.24e-05    2.79   0.0053 ** 
## I(kilometer^2)     -2.57e-10   1.51e-11  -17.09   <2e-16 ***
## I(log(kilometer))   1.92e-01   2.08e-01    0.92   0.3555    
## ---
## Signif. codes:  0 '***' 0.001 '**' 0.01 '*' 0.05 '.' 0.1 ' ' 1
## 
## Residual standard error: 2.5 on 1927 degrees of freedom
## Multiple R-squared:  0.945,  Adjusted R-squared:  0.945 
## F-statistic: 6.6e+03 on 5 and 1927 DF,  p-value: <2e-16
\end{verbatim}

\begin{Shaded}
\begin{Highlighting}[]
\CommentTok{#(coef(int_model) - coef(int_model_sub)) / coef(int_model)}
\KeywordTok{shapiro.test}\NormalTok{(}\KeywordTok{resid}\NormalTok{(model_cox_sub))}
\end{Highlighting}
\end{Shaded}

\begin{verbatim}
## 
##  Shapiro-Wilk normality test
## 
## data:  resid(model_cox_sub)
## W = 1, p-value = 0.0004
\end{verbatim}

\begin{Shaded}
\begin{Highlighting}[]
\KeywordTok{bptest}\NormalTok{(model_cox_sub)}
\end{Highlighting}
\end{Shaded}

\begin{verbatim}
## 
##  studentized Breusch-Pagan test
## 
## data:  model_cox_sub
## BP = 133, df = 5, p-value <2e-16
\end{verbatim}

\hypertarget{determine-best-lambda-for-box-cox-tranformation}{%
\subsection{2.1.4 determine best lambda for box-cox
tranformation}\label{determine-best-lambda-for-box-cox-tranformation}}

We will run box-cox transformation on subgroup of data that has more
than 300 records..

\begin{Shaded}
\begin{Highlighting}[]
\KeywordTok{source}\NormalTok{(}\StringTok{"misc_functions.R"}\NormalTok{)}
\NormalTok{(}\DataTypeTok{boxcox_lambda=}\KeywordTok{subset_autodata_with_boxcox}\NormalTok{(autos))}
\KeywordTok{hist}\NormalTok{(boxcox_lambda,}\DataTypeTok{breaks=}\DecValTok{20}\NormalTok{,}\DataTypeTok{col=}\StringTok{"lightblue"}\NormalTok{)}
\KeywordTok{mean}\NormalTok{(boxcox_lambda)}
\end{Highlighting}
\end{Shaded}

Based on our analysis above, we will use \(\lambda=0.3\) for the Box-Cox
transformation!

\begin{Shaded}
\begin{Highlighting}[]
\NormalTok{lambda=}\FloatTok{0.3}
\NormalTok{formula_str=}\KeywordTok{as.character}\NormalTok{(model_cox}\OperatorTok{$}\NormalTok{call[}\DecValTok{2}\NormalTok{])}
\end{Highlighting}
\end{Shaded}

\hypertarget{phase-ii---adding-categorical-variable-to-the-model-form-determine-in-phase-i}{%
\subsubsection{Phase II - adding categorical variable to the model form
determine in Phase
I}\label{phase-ii---adding-categorical-variable-to-the-model-form-determine-in-phase-i}}

\hypertarget{the-best-model-form-with-only-continous-variable-is}{%
\subparagraph{The best model form with only continous variable
is:}\label{the-best-model-form-with-only-continous-variable-is}}

(price\^{}lambda - 1)/lambda \textasciitilde{} powerPS +
yearOfRegistration + I(powerPS\^{}2) + I(kilometer\^{}2) +
I(log(kilometer))

\hypertarget{next-we-combine-continous-variableswiht-established-form-with-all-factor-varible-to-start-a-backward-aic-to-find-a-good-model.}{%
\paragraph{Next, we combine continous variables(wiht established form)
with all factor varible to start a backward AIC to find a good
model.}\label{next-we-combine-continous-variableswiht-established-form-with-all-factor-varible-to-start-a-backward-aic-to-find-a-good-model.}}

\begin{Shaded}
\begin{Highlighting}[]
\NormalTok{size_train=}\DecValTok{2000}

\NormalTok{idx_train=}\KeywordTok{sample}\NormalTok{(}\DecValTok{1}\OperatorTok{:}\KeywordTok{nrow}\NormalTok{(autos),size_train)}
\NormalTok{autos_train=autos[idx_train,]}
\NormalTok{autos_train[,columns_factor]=}\KeywordTok{lapply}\NormalTok{(autos_train[,columns_factor], as.factor)}
\CommentTok{#model2_start = lm( ((price^lambda - 1)/lambda) ~ .^2 +  I(powerPS^2) + I(kilometer^2) + I(log(kilometer)), autos_train ) }
\NormalTok{model2_start =}\StringTok{ }\KeywordTok{lm}\NormalTok{( ((price}\OperatorTok{^}\NormalTok{lambda }\OperatorTok{-}\StringTok{ }\DecValTok{1}\NormalTok{)}\OperatorTok{/}\NormalTok{lambda) }\OperatorTok{~}\StringTok{ }\NormalTok{. }\OperatorTok{+}\StringTok{  }\KeywordTok{I}\NormalTok{(powerPS}\OperatorTok{^}\DecValTok{2}\NormalTok{) }\OperatorTok{+}\StringTok{ }\KeywordTok{I}\NormalTok{(kilometer}\OperatorTok{^}\DecValTok{2}\NormalTok{) }\OperatorTok{+}\StringTok{ }\KeywordTok{I}\NormalTok{(}\KeywordTok{log}\NormalTok{(kilometer)), }\DataTypeTok{data=}\NormalTok{autos_train ) }
\NormalTok{n=}\KeywordTok{nrow}\NormalTok{(autos_train)}
\NormalTok{model2_selected_bic =}\StringTok{ }\KeywordTok{step}\NormalTok{(model2_start,}\DataTypeTok{k=}\KeywordTok{log}\NormalTok{(n),}\DataTypeTok{trace=}\DecValTok{0}\NormalTok{)}
\NormalTok{model2_selected_bic}
\end{Highlighting}
\end{Shaded}

\begin{verbatim}
## 
## Call:
## lm(formula = ((price^lambda - 1)/lambda) ~ vehicleType + yearOfRegistration + 
##     gearbox + powerPS + kilometer + brand + notRepairedDamage + 
##     I(powerPS^2) + I(kilometer^2) + I(log(kilometer)), data = autos_train)
## 
## Coefficients:
##           (Intercept)         vehicleTypebus      vehicleTypecabrio  
##             -3.64e+03               3.42e+00               6.39e+00  
##      vehicleTypecoupe  vehicleTypekleinwagen       vehicleTypekombi  
##              3.25e+00               1.72e-01               8.48e-01  
##  vehicleTypelimousine         vehicleTypesuv     yearOfRegistration  
##              1.53e+00               7.17e+00               1.85e+00  
##        gearboxmanuell                powerPS              kilometer  
##             -1.60e+00               7.55e-02               1.46e-04  
##             brandaudi               brandbmw         brandchevrolet  
##              6.71e+00               6.17e+00              -2.39e+00  
##         brandchrysler           brandcitroen             branddacia  
##              2.17e+00              -1.95e+00              -5.17e+00  
##         branddaihatsu              brandfiat              brandford  
##              2.75e+00              -1.10e+00              -6.12e-01  
##            brandhonda           brandhyundai            brandjaguar  
##              7.44e-01              -2.26e+00               5.01e+00  
##             brandjeep               brandkia              brandlada  
##              8.81e-01              -1.67e+00              -8.76e+00  
##           brandlancia        brandland_rover             brandmazda  
##             -3.15e+00               5.50e+00               1.82e+00  
##    brandmercedes_benz              brandmini        brandmitsubishi  
##              6.10e+00               5.23e+00               1.19e+00  
##           brandnissan              brandopel           brandpeugeot  
##              1.57e+00               5.52e-01              -1.49e+00  
##          brandporsche           brandrenault             brandrover  
##              1.79e+01              -8.65e-01               1.22e+00  
##             brandsaab              brandseat             brandskoda  
##             -2.66e+00               1.13e+00               2.62e+00  
##            brandsmart            brandsubaru            brandsuzuki  
##             -1.89e+00               4.57e+00               2.51e+00  
##           brandtoyota        brandvolkswagen             brandvolvo  
##              1.38e+00               4.73e+00               3.59e+00  
## notRepairedDamagenein           I(powerPS^2)         I(kilometer^2)  
##              4.99e+00              -4.31e-05              -6.67e-10  
##     I(log(kilometer))  
##             -4.01e+00
\end{verbatim}

\hypertarget{remove-high-influence-data-and-refit-the-choose-model}{%
\subsection{2.1.4 remove high influence data and refit the choose
model}\label{remove-high-influence-data-and-refit-the-choose-model}}

\begin{Shaded}
\begin{Highlighting}[]
\NormalTok{model =model2_selected_bic}
\KeywordTok{par}\NormalTok{(}\DataTypeTok{mfrow=}\KeywordTok{c}\NormalTok{(}\DecValTok{1}\NormalTok{,}\DecValTok{2}\NormalTok{))}
\KeywordTok{plot}\NormalTok{(}\KeywordTok{fitted}\NormalTok{(model), }\KeywordTok{resid}\NormalTok{(model), }\DataTypeTok{col =} \StringTok{"dodgerblue"}\NormalTok{, }
     \DataTypeTok{xlab =} \StringTok{"Fitted"}\NormalTok{, }\DataTypeTok{ylab =} \StringTok{"Residuals"}\NormalTok{, }\DataTypeTok{main =} \StringTok{"Fitted versus Residuals"}\NormalTok{)}
\KeywordTok{abline}\NormalTok{(}\DataTypeTok{h =} \DecValTok{0}\NormalTok{, }\DataTypeTok{col =} \StringTok{"darkorange"}\NormalTok{, }\DataTypeTok{lwd =} \DecValTok{2}\NormalTok{)}
\KeywordTok{qqnorm}\NormalTok{(}\KeywordTok{resid}\NormalTok{(model), }\DataTypeTok{main =} \StringTok{"Normal Q-Q Plot"}\NormalTok{, }\DataTypeTok{col =} \StringTok{"dodgerblue"}\NormalTok{)}
\KeywordTok{qqline}\NormalTok{(}\KeywordTok{resid}\NormalTok{(model), }\DataTypeTok{col =} \StringTok{"dodgerblue"}\NormalTok{, }\DataTypeTok{lwd =} \DecValTok{2}\NormalTok{)}
\end{Highlighting}
\end{Shaded}

\includegraphics{team-a-final-proj_files/figure-latex/unnamed-chunk-10-1.pdf}

\begin{Shaded}
\begin{Highlighting}[]
\KeywordTok{shapiro.test}\NormalTok{(}\KeywordTok{resid}\NormalTok{(model))}
\end{Highlighting}
\end{Shaded}

\begin{verbatim}
## 
##  Shapiro-Wilk normality test
## 
## data:  resid(model)
## W = 0.93, p-value <2e-16
\end{verbatim}

\begin{Shaded}
\begin{Highlighting}[]
\KeywordTok{bptest}\NormalTok{(model)}
\end{Highlighting}
\end{Shaded}

\begin{verbatim}
## 
##  studentized Breusch-Pagan test
## 
## data:  model
## BP = 62, df = 51, p-value = 0.1
\end{verbatim}

\begin{Shaded}
\begin{Highlighting}[]
\CommentTok{#finding influenctial}
\NormalTok{mod_cook =}\StringTok{ }\KeywordTok{cooks.distance}\NormalTok{(model)}
\NormalTok{n=}\KeywordTok{length}\NormalTok{(}\KeywordTok{resid}\NormalTok{(model))}
\NormalTok{high_infl =}\StringTok{ }\NormalTok{mod_cook }\OperatorTok{>}\StringTok{ }\DecValTok{4} \OperatorTok{/}\StringTok{ }\NormalTok{n}
\KeywordTok{sum}\NormalTok{(high_infl)     }
\end{Highlighting}
\end{Shaded}

\begin{verbatim}
## [1] NA
\end{verbatim}

\begin{Shaded}
\begin{Highlighting}[]
\KeywordTok{mean}\NormalTok{(high_infl)}
\end{Highlighting}
\end{Shaded}

\begin{verbatim}
## [1] NA
\end{verbatim}

\begin{Shaded}
\begin{Highlighting}[]
\CommentTok{#Refit the multiple regression model without any influential points}
\NormalTok{formula=}\KeywordTok{as.formula}\NormalTok{(}\KeywordTok{as.character}\NormalTok{(model}\OperatorTok{$}\NormalTok{call[}\DecValTok{2}\NormalTok{]))}
\NormalTok{model_sub =}\StringTok{ }\KeywordTok{lm}\NormalTok{(formula, }\DataTypeTok{data =}\NormalTok{ autos_train, }\DataTypeTok{subset =} \OperatorTok{!}\NormalTok{high_infl)}
\KeywordTok{par}\NormalTok{(}\DataTypeTok{mfrow=}\KeywordTok{c}\NormalTok{(}\DecValTok{1}\NormalTok{,}\DecValTok{2}\NormalTok{))}
\KeywordTok{plot}\NormalTok{(}\KeywordTok{fitted}\NormalTok{(model_sub), }\KeywordTok{resid}\NormalTok{(model_sub), }\DataTypeTok{col =} \StringTok{"dodgerblue"}\NormalTok{, }
     \DataTypeTok{xlab =} \StringTok{"Fitted"}\NormalTok{, }\DataTypeTok{ylab =} \StringTok{"Residuals"}\NormalTok{, }\DataTypeTok{main =} \StringTok{"Fitted versus Residuals"}\NormalTok{)}
\KeywordTok{abline}\NormalTok{(}\DataTypeTok{h =} \DecValTok{0}\NormalTok{, }\DataTypeTok{col =} \StringTok{"darkorange"}\NormalTok{, }\DataTypeTok{lwd =} \DecValTok{2}\NormalTok{)}
\KeywordTok{qqnorm}\NormalTok{(}\KeywordTok{resid}\NormalTok{(model_sub), }\DataTypeTok{main =} \StringTok{"Normal Q-Q Plot"}\NormalTok{, }\DataTypeTok{col =} \StringTok{"dodgerblue"}\NormalTok{)}
\KeywordTok{qqline}\NormalTok{(}\KeywordTok{resid}\NormalTok{(model_sub), }\DataTypeTok{col =} \StringTok{"dodgerblue"}\NormalTok{, }\DataTypeTok{lwd =} \DecValTok{2}\NormalTok{)}
\end{Highlighting}
\end{Shaded}

\includegraphics{team-a-final-proj_files/figure-latex/unnamed-chunk-10-2.pdf}

\begin{Shaded}
\begin{Highlighting}[]
\KeywordTok{shapiro.test}\NormalTok{(}\KeywordTok{resid}\NormalTok{(model_sub))}
\end{Highlighting}
\end{Shaded}

\begin{verbatim}
## 
##  Shapiro-Wilk normality test
## 
## data:  resid(model_sub)
## W = 1, p-value = 7e-06
\end{verbatim}

\begin{Shaded}
\begin{Highlighting}[]
\KeywordTok{bptest}\NormalTok{(model_sub)}
\end{Highlighting}
\end{Shaded}

\begin{verbatim}
## 
##  studentized Breusch-Pagan test
## 
## data:  model_sub
## BP = 121, df = 48, p-value = 3e-08
\end{verbatim}

\begin{Shaded}
\begin{Highlighting}[]
\KeywordTok{summary}\NormalTok{(model_sub)}\OperatorTok{$}\NormalTok{adj.r.squared}
\end{Highlighting}
\end{Shaded}

\begin{verbatim}
## [1] 0.8996
\end{verbatim}

\begin{Shaded}
\begin{Highlighting}[]
\KeywordTok{coef}\NormalTok{(}\KeywordTok{summary}\NormalTok{(model_sub))}
\end{Highlighting}
\end{Shaded}

\begin{verbatim}
##                         Estimate Std. Error  t value  Pr(>|t|)
## (Intercept)           -3.631e+03  6.452e+01 -56.2808 0.000e+00
## vehicleTypebus         2.817e+00  1.424e+00   1.9791 4.795e-02
## vehicleTypecabrio      4.873e+00  1.457e+00   3.3434 8.442e-04
## vehicleTypecoupe       2.249e+00  1.463e+00   1.5374 1.244e-01
## vehicleTypekleinwagen -3.570e-01  1.410e+00  -0.2532 8.002e-01
## vehicleTypekombi       2.637e-01  1.415e+00   0.1864 8.522e-01
## vehicleTypelimousine   6.777e-01  1.412e+00   0.4798 6.314e-01
## vehicleTypesuv         6.185e+00  1.477e+00   4.1871 2.958e-05
## yearOfRegistration     1.843e+00  3.123e-02  59.0198 0.000e+00
## gearboxmanuell        -1.451e+00  2.592e-01  -5.5990 2.480e-08
## powerPS                7.100e-02  5.377e-03  13.2051 4.108e-38
## kilometer              1.546e-04  4.690e-05   3.2959 9.998e-04
## brandaudi              6.755e+00  1.178e+00   5.7322 1.155e-08
## brandbmw               6.784e+00  1.168e+00   5.8074 7.453e-09
## brandchevrolet        -3.286e+00  1.754e+00  -1.8740 6.108e-02
## brandchrysler          1.195e+00  2.560e+00   0.4668 6.407e-01
## brandcitroen          -1.845e+00  1.368e+00  -1.3483 1.777e-01
## branddacia            -5.536e+00  1.885e+00  -2.9364 3.361e-03
## branddaihatsu          1.982e+00  4.112e+00   0.4820 6.299e-01
## brandfiat             -8.252e-01  1.268e+00  -0.6506 5.154e-01
## brandford             -6.581e-01  1.187e+00  -0.5542 5.795e-01
## brandhonda             8.843e-01  1.880e+00   0.4704 6.382e-01
## brandhyundai          -2.268e+00  1.410e+00  -1.6082 1.080e-01
## brandjaguar            3.477e+00  4.173e+00   0.8332 4.049e-01
## brandjeep              2.303e+00  2.350e+00   0.9803 3.271e-01
## brandkia              -2.044e+00  1.513e+00  -1.3509 1.769e-01
## brandlada             -9.481e+00  3.055e+00  -3.1034 1.942e-03
## brandland_rover        5.543e+00  2.586e+00   2.1433 3.222e-02
## brandmazda             2.255e+00  1.379e+00   1.6350 1.022e-01
## brandmercedes_benz     5.603e+00  1.179e+00   4.7505 2.186e-06
## brandmini              5.572e+00  1.432e+00   3.8915 1.032e-04
## brandmitsubishi        1.001e+00  1.693e+00   0.5913 5.544e-01
## brandnissan            8.648e-01  1.407e+00   0.6147 5.388e-01
## brandopel              7.021e-01  1.178e+00   0.5960 5.512e-01
## brandpeugeot          -1.193e+00  1.232e+00  -0.9681 3.331e-01
## brandporsche           1.646e+01  1.704e+00   9.6609 1.415e-21
## brandrenault          -8.178e-01  1.230e+00  -0.6647 5.063e-01
## brandsaab             -1.204e+00  3.015e+00  -0.3993 6.897e-01
## brandseat              1.575e+00  1.277e+00   1.2337 2.175e-01
## brandskoda             2.653e+00  1.265e+00   2.0980 3.604e-02
## brandsmart            -1.399e+00  1.376e+00  -1.0166 3.095e-01
## brandsuzuki            3.234e+00  1.810e+00   1.7868 7.413e-02
## brandtoyota            1.069e+00  1.327e+00   0.8056 4.206e-01
## brandvolkswagen        4.761e+00  1.159e+00   4.1079 4.167e-05
## brandvolvo             3.953e+00  1.648e+00   2.3989 1.654e-02
## notRepairedDamagenein  4.741e+00  3.431e-01  13.8182 2.098e-41
## I(powerPS^2)          -1.685e-05  1.311e-05  -1.2850 1.990e-01
## I(kilometer^2)        -7.109e-10  1.584e-10  -4.4887 7.608e-06
## I(log(kilometer))     -4.112e+00  1.379e+00  -2.9815 2.906e-03
\end{verbatim}

\begin{Shaded}
\begin{Highlighting}[]
\NormalTok{calc_loocv_rmse =}\StringTok{ }\ControlFlowTok{function}\NormalTok{(model) \{}
  \KeywordTok{sqrt}\NormalTok{(}\KeywordTok{mean}\NormalTok{((}\KeywordTok{resid}\NormalTok{(model) }\OperatorTok{/}\StringTok{ }\NormalTok{(}\DecValTok{1} \OperatorTok{-}\StringTok{ }\KeywordTok{hatvalues}\NormalTok{(model))) }\OperatorTok{^}\StringTok{ }\DecValTok{2}\NormalTok{))}
\NormalTok{\}}
\end{Highlighting}
\end{Shaded}

\hypertarget{test-results}{%
\subsection{test results}\label{test-results}}

TEST result?

LOOC\_RMSE CV

\hypertarget{results}{%
\section{3. Results}\label{results}}

\hypertarget{discussion}{%
\section{4. Discussion}\label{discussion}}

\hypertarget{appendix}{%
\section{5. Appendix}\label{appendix}}


\end{document}
