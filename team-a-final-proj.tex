\documentclass[]{article}
\usepackage{lmodern}
\usepackage{amssymb,amsmath}
\usepackage{ifxetex,ifluatex}
\usepackage{fixltx2e} % provides \textsubscript
\ifnum 0\ifxetex 1\fi\ifluatex 1\fi=0 % if pdftex
  \usepackage[T1]{fontenc}
  \usepackage[utf8]{inputenc}
\else % if luatex or xelatex
  \ifxetex
    \usepackage{mathspec}
  \else
    \usepackage{fontspec}
  \fi
  \defaultfontfeatures{Ligatures=TeX,Scale=MatchLowercase}
\fi
% use upquote if available, for straight quotes in verbatim environments
\IfFileExists{upquote.sty}{\usepackage{upquote}}{}
% use microtype if available
\IfFileExists{microtype.sty}{%
\usepackage{microtype}
\UseMicrotypeSet[protrusion]{basicmath} % disable protrusion for tt fonts
}{}
\usepackage[margin=1in]{geometry}
\usepackage{hyperref}
\PassOptionsToPackage{usenames,dvipsnames}{color} % color is loaded by hyperref
\hypersetup{unicode=true,
            pdftitle={Building an Used Car Price Prediction Model for Germany eBay listings},
            pdfauthor={Richa Gupta/Sandro Tanis/Ping Wang},
            colorlinks=true,
            linkcolor=Maroon,
            citecolor=Blue,
            urlcolor=cyan,
            breaklinks=true}
\urlstyle{same}  % don't use monospace font for urls
\usepackage{color}
\usepackage{fancyvrb}
\newcommand{\VerbBar}{|}
\newcommand{\VERB}{\Verb[commandchars=\\\{\}]}
\DefineVerbatimEnvironment{Highlighting}{Verbatim}{commandchars=\\\{\}}
% Add ',fontsize=\small' for more characters per line
\usepackage{framed}
\definecolor{shadecolor}{RGB}{248,248,248}
\newenvironment{Shaded}{\begin{snugshade}}{\end{snugshade}}
\newcommand{\AlertTok}[1]{\textcolor[rgb]{0.94,0.16,0.16}{#1}}
\newcommand{\AnnotationTok}[1]{\textcolor[rgb]{0.56,0.35,0.01}{\textbf{\textit{#1}}}}
\newcommand{\AttributeTok}[1]{\textcolor[rgb]{0.77,0.63,0.00}{#1}}
\newcommand{\BaseNTok}[1]{\textcolor[rgb]{0.00,0.00,0.81}{#1}}
\newcommand{\BuiltInTok}[1]{#1}
\newcommand{\CharTok}[1]{\textcolor[rgb]{0.31,0.60,0.02}{#1}}
\newcommand{\CommentTok}[1]{\textcolor[rgb]{0.56,0.35,0.01}{\textit{#1}}}
\newcommand{\CommentVarTok}[1]{\textcolor[rgb]{0.56,0.35,0.01}{\textbf{\textit{#1}}}}
\newcommand{\ConstantTok}[1]{\textcolor[rgb]{0.00,0.00,0.00}{#1}}
\newcommand{\ControlFlowTok}[1]{\textcolor[rgb]{0.13,0.29,0.53}{\textbf{#1}}}
\newcommand{\DataTypeTok}[1]{\textcolor[rgb]{0.13,0.29,0.53}{#1}}
\newcommand{\DecValTok}[1]{\textcolor[rgb]{0.00,0.00,0.81}{#1}}
\newcommand{\DocumentationTok}[1]{\textcolor[rgb]{0.56,0.35,0.01}{\textbf{\textit{#1}}}}
\newcommand{\ErrorTok}[1]{\textcolor[rgb]{0.64,0.00,0.00}{\textbf{#1}}}
\newcommand{\ExtensionTok}[1]{#1}
\newcommand{\FloatTok}[1]{\textcolor[rgb]{0.00,0.00,0.81}{#1}}
\newcommand{\FunctionTok}[1]{\textcolor[rgb]{0.00,0.00,0.00}{#1}}
\newcommand{\ImportTok}[1]{#1}
\newcommand{\InformationTok}[1]{\textcolor[rgb]{0.56,0.35,0.01}{\textbf{\textit{#1}}}}
\newcommand{\KeywordTok}[1]{\textcolor[rgb]{0.13,0.29,0.53}{\textbf{#1}}}
\newcommand{\NormalTok}[1]{#1}
\newcommand{\OperatorTok}[1]{\textcolor[rgb]{0.81,0.36,0.00}{\textbf{#1}}}
\newcommand{\OtherTok}[1]{\textcolor[rgb]{0.56,0.35,0.01}{#1}}
\newcommand{\PreprocessorTok}[1]{\textcolor[rgb]{0.56,0.35,0.01}{\textit{#1}}}
\newcommand{\RegionMarkerTok}[1]{#1}
\newcommand{\SpecialCharTok}[1]{\textcolor[rgb]{0.00,0.00,0.00}{#1}}
\newcommand{\SpecialStringTok}[1]{\textcolor[rgb]{0.31,0.60,0.02}{#1}}
\newcommand{\StringTok}[1]{\textcolor[rgb]{0.31,0.60,0.02}{#1}}
\newcommand{\VariableTok}[1]{\textcolor[rgb]{0.00,0.00,0.00}{#1}}
\newcommand{\VerbatimStringTok}[1]{\textcolor[rgb]{0.31,0.60,0.02}{#1}}
\newcommand{\WarningTok}[1]{\textcolor[rgb]{0.56,0.35,0.01}{\textbf{\textit{#1}}}}
\usepackage{longtable,booktabs}
\usepackage{graphicx,grffile}
\makeatletter
\def\maxwidth{\ifdim\Gin@nat@width>\linewidth\linewidth\else\Gin@nat@width\fi}
\def\maxheight{\ifdim\Gin@nat@height>\textheight\textheight\else\Gin@nat@height\fi}
\makeatother
% Scale images if necessary, so that they will not overflow the page
% margins by default, and it is still possible to overwrite the defaults
% using explicit options in \includegraphics[width, height, ...]{}
\setkeys{Gin}{width=\maxwidth,height=\maxheight,keepaspectratio}
\IfFileExists{parskip.sty}{%
\usepackage{parskip}
}{% else
\setlength{\parindent}{0pt}
\setlength{\parskip}{6pt plus 2pt minus 1pt}
}
\setlength{\emergencystretch}{3em}  % prevent overfull lines
\providecommand{\tightlist}{%
  \setlength{\itemsep}{0pt}\setlength{\parskip}{0pt}}
\setcounter{secnumdepth}{0}
% Redefines (sub)paragraphs to behave more like sections
\ifx\paragraph\undefined\else
\let\oldparagraph\paragraph
\renewcommand{\paragraph}[1]{\oldparagraph{#1}\mbox{}}
\fi
\ifx\subparagraph\undefined\else
\let\oldsubparagraph\subparagraph
\renewcommand{\subparagraph}[1]{\oldsubparagraph{#1}\mbox{}}
\fi

%%% Use protect on footnotes to avoid problems with footnotes in titles
\let\rmarkdownfootnote\footnote%
\def\footnote{\protect\rmarkdownfootnote}

%%% Change title format to be more compact
\usepackage{titling}

% Create subtitle command for use in maketitle
\providecommand{\subtitle}[1]{
  \posttitle{
    \begin{center}\large#1\end{center}
    }
}

\setlength{\droptitle}{-2em}

  \title{Building an Used Car Price Prediction Model for Germany eBay listings}
    \pretitle{\vspace{\droptitle}\centering\huge}
  \posttitle{\par}
    \author{Richa Gupta/Sandro Tanis/Ping Wang}
    \preauthor{\centering\large\emph}
  \postauthor{\par}
      \predate{\centering\large\emph}
  \postdate{\par}
    \date{Aug 07, 2020}


\begin{document}
\maketitle

\begin{center}\rule{0.5\linewidth}{\linethickness}\end{center}

\hypertarget{introduction}{%
\section{1. Introduction}\label{introduction}}

Why this Topic?

The formation of our group hinged upon our interest in the automotive
industry, as a result of that we have chosen this robust dataset because
we wanted to have a better understanding about the used car market and
this dataset had all of the components that we wanted to observe for
this project.The dataset that I am using in this project was found on
Kaggle, the well-known Machine Learning Competition website and it is
about Used Car listings from eBay -- Germany.

The Data file contains approximately 370,000 observations and 20
variables that are scraped from used-car listing on Ebay-Kleinanzeigen
(German). The high quantity and authenticity make this dataset useful
for an exploratory analysis. Furthermore, the dataset is very large but
what we have done to narrow it down is that we have decided to filter
the data based on the year registration variable starting the year 2000
to now so we can build a good model based on the last 20 years in the
used auto industry in Germany. Some of the of the variables that are
very useful tous where price is used as a response to help build model
predication when combined with other variables but not limited to :
Power, Kilometer, VehicleType, YearRegistration as part of our additive
model for this project as you will find out that based on our research
we were able to determine whichmodel is sufficient for us to use based
on these continuous variables in the dataset.

This project focuses on the exploratory data analysis phase of the
dataset. In particular, we will try to detect associations between
variables, especially against price. The end-goal of such a project
would be to build a price-prediction model for vehicles sold by eBay
users. This project broke down a lot of the pure concepts that allowed
us to bring together the concepts we learned as well as provide
background information to understand how these models are determined.

\hypertarget{methods}{%
\section{2. Methods}\label{methods}}

\hypertarget{data-preparation}{%
\subsection{2.0 Data preparation}\label{data-preparation}}

\begin{Shaded}
\begin{Highlighting}[]
\KeywordTok{library}\NormalTok{(dplyr)}
\KeywordTok{library}\NormalTok{(readr)}
\KeywordTok{library}\NormalTok{(lmtest)}
\KeywordTok{library}\NormalTok{(MASS)}
\KeywordTok{source}\NormalTok{(}\StringTok{"misc_functions.R"}\NormalTok{)}
\end{Highlighting}
\end{Shaded}

The major libraries needed will be loaded from here to be able to be
used throughout the document. This will include the libraries such as
LMTEST to test linear regression model, read input from a csv files
readr, and dplyr which enables us to manipulate the dataset.

\hypertarget{data-loading}{%
\subsubsection{2.0.1 Data loading}\label{data-loading}}

Loading the raw data from the CSV file:

\begin{Shaded}
\begin{Highlighting}[]
\NormalTok{autos_raw <-}\StringTok{ }\KeywordTok{read_csv}\NormalTok{(}\StringTok{"autos.csv"}\NormalTok{)}
\KeywordTok{attr}\NormalTok{(autos_raw, }\StringTok{'spec'}\NormalTok{) <-}\StringTok{ }\OtherTok{NULL}
\KeywordTok{attr}\NormalTok{(autos_raw, }\StringTok{'problems'}\NormalTok{) <-}\StringTok{ }\OtherTok{NULL}

\KeywordTok{str}\NormalTok{(autos_raw)}
\end{Highlighting}
\end{Shaded}

\begin{verbatim}
## Classes 'spec_tbl_df', 'tbl_df', 'tbl' and 'data.frame': 354687 obs. of  20 variables:
##  $ dateCrawled        : POSIXct, format: "2016-03-24 11:52:17" "2016-03-24 10:58:45" ...
##  $ name               : chr  "Golf_3_1.6" "A5_Sportback_2.7_Tdi" "Jeep_Grand_Cherokee_\"Overland\"" "GOLF_4_1_4__3T\xdcRER" ...
##  $ seller             : chr  "privat" "privat" "privat" "privat" ...
##  $ offerType          : chr  "Angebot" "Angebot" "Angebot" "Angebot" ...
##  $ price              : num  480 18300 9800 1500 3600 650 2200 0 14500 999 ...
##  $ abtest             : chr  "test" "test" "test" "test" ...
##  $ vehicleType        : chr  NA "coupe" "suv" "kleinwagen" ...
##  $ yearOfRegistration : num  1993 2011 2004 2001 2008 ...
##  $ gearbox            : chr  "manuell" "manuell" "automatik" "manuell" ...
##  $ powerPS            : num  0 190 163 75 69 102 109 50 125 101 ...
##  $ model              : chr  "golf" NA "grand" "golf" ...
##  $ kilometer          : num  150000 125000 125000 150000 90000 150000 150000 40000 30000 150000 ...
##  $ monthOfRegistration: num  0 5 8 6 7 10 8 7 8 0 ...
##  $ fuelType           : chr  "benzin" "diesel" "diesel" "benzin" ...
##  $ brand              : chr  "volkswagen" "audi" "jeep" "volkswagen" ...
##  $ notRepairedDamage  : chr  NA "ja" NA "nein" ...
##  $ dateCreated        : POSIXct, format: "2016-03-24" "2016-03-24" ...
##  $ nrOfPictures       : num  0 0 0 0 0 0 0 0 0 0 ...
##  $ postalCode         : chr  "70435" "66954" "90480" "91074" ...
##  $ lastSeen           : POSIXct, format: "2016-04-07 03:16:57" "2016-04-07 01:46:50" ...
\end{verbatim}

\hypertarget{data-cleaning-and-unused-columns-removal}{%
\subsubsection{2.0.2 Data cleaning and unused columns
removal}\label{data-cleaning-and-unused-columns-removal}}

\begin{Shaded}
\begin{Highlighting}[]
\NormalTok{autos=}\KeywordTok{subset}\NormalTok{(autos_raw, price}\OperatorTok{>}\DecValTok{500} \OperatorTok{&}\StringTok{ }\NormalTok{price}\OperatorTok{<}\DecValTok{200000} \OperatorTok{&}\StringTok{ }\NormalTok{yearOfRegistration}\OperatorTok{>=}\DecValTok{2000} \OperatorTok{&}\StringTok{ }\NormalTok{powerPS}\OperatorTok{>}\DecValTok{0} \OperatorTok{&}\StringTok{ }\NormalTok{offerType}\OperatorTok{==}\StringTok{"Angebot"} \OperatorTok{&}\StringTok{ }\NormalTok{seller}\OperatorTok{==}\StringTok{"privat"}\NormalTok{)}
\NormalTok{columns_remove=}\KeywordTok{c}\NormalTok{(}\StringTok{"postalCode"}\NormalTok{,}\StringTok{"lastSeen"}\NormalTok{,}\StringTok{"name"}\NormalTok{,}\StringTok{"model"}\NormalTok{,}\StringTok{"brand"}\NormalTok{, }\StringTok{"nrOfPictures"}\NormalTok{,}\StringTok{"dateCreated"}\NormalTok{,}\StringTok{"dateCrawled"}\NormalTok{,}\StringTok{"monthOfRegistration"}\NormalTok{,}\StringTok{"offerType"}\NormalTok{,}\StringTok{"seller"}\NormalTok{)   }
\NormalTok{columns_numeric =}\StringTok{ }\KeywordTok{c}\NormalTok{(}\StringTok{"price"}\NormalTok{,}\StringTok{"powerPS"}\NormalTok{,}\StringTok{"yearOfRegistration"}\NormalTok{,}\StringTok{"kilometer"}\NormalTok{)}
\NormalTok{columns_factor =}\StringTok{ }\KeywordTok{c}\NormalTok{(}\StringTok{"abtest"}\NormalTok{,}\StringTok{"vehicleType"}\NormalTok{,}\StringTok{"gearbox"}\NormalTok{,}\StringTok{"fuelType"}\NormalTok{,}\StringTok{"notRepairedDamage"}\NormalTok{ )}
\NormalTok{autos=}\KeywordTok{na.omit}\NormalTok{(autos)}
\NormalTok{autos =}\StringTok{ }\NormalTok{autos[, }\OperatorTok{-}\KeywordTok{which}\NormalTok{(}\KeywordTok{names}\NormalTok{(autos) }\OperatorTok\StringTok{ }\NormalTok{columns_remove)]}
\end{Highlighting}
\end{Shaded}

Our data clearning process involved removing unuseful variable such as
postalCode, lastSeen, name ect.. and minize to some of the useful
variable that we could use which includes the numerical and factor
variables.

After loading the raw data, \textbf{autos\_raw} we:

\begin{itemize}
\tightlist
\item
  Removed data with \(price < 0\)
\item
  Keep only data with offerType=``Angebot'' and seller==``privat''.
  **The data can either be in the form offertype = ``Anagebot'' and
  seller =``private''
\item
  Keep only data with YearOfRegistration\textgreater{}=2000, **We
  narrowed our dataset to include the last 20 years of used cars
  starting with the year 2000 and beyond
\item
  Removed unused columns: postalCode, lastSeen, name, model, brand,
  nrOfPictures, dateCreated, dateCrawled, monthOfRegistration,
  offerType, seller, **We removed unused/unwanted columns in our
  cleaning process
\item
  Indentify continous numeric columns : price, powerPS,
  yearOfRegistration, kilometer
\item
  Indentify factor columns : abtest, vehicleType, gearbox, fuelType,
  notRepairedDamage
\end{itemize}

\begin{Shaded}
\begin{Highlighting}[]
\CommentTok{#final data format}
\KeywordTok{str}\NormalTok{(autos)}
\end{Highlighting}
\end{Shaded}

\begin{verbatim}
## Classes 'tbl_df', 'tbl' and 'data.frame':    176516 obs. of  9 variables:
##  $ price             : num  1500 3600 2200 2000 2799 ...
##  $ abtest            : chr  "test" "test" "test" "control" ...
##  $ vehicleType       : chr  "kleinwagen" "kleinwagen" "cabrio" "limousine" ...
##  $ yearOfRegistration: num  2001 2008 2004 2004 2005 ...
##  $ gearbox           : chr  "manuell" "manuell" "manuell" "manuell" ...
##  $ powerPS           : num  75 69 109 105 140 190 75 136 102 160 ...
##  $ kilometer         : num  150000 90000 150000 150000 150000 70000 150000 150000 150000 100000 ...
##  $ fuelType          : chr  "benzin" "diesel" "benzin" "benzin" ...
##  $ notRepairedDamage : chr  "nein" "nein" "nein" "nein" ...
##  - attr(*, "na.action")= 'omit' Named int  1 2 6 9 16 18 21 25 26 28 ...
##   ..- attr(*, "names")= chr  "1" "2" "6" "9" ...
\end{verbatim}

Taking a look at the continous variables only - checking for colinearity

\begin{Shaded}
\begin{Highlighting}[]
\CommentTok{#Checking colinear on numberic columns on 50000 records sample}
\NormalTok{sample_size=}\DecValTok{50000}
\NormalTok{idx_sample=}\KeywordTok{sample}\NormalTok{(}\DecValTok{1}\OperatorTok{:}\KeywordTok{nrow}\NormalTok{(autos),sample_size)}
\NormalTok{autos_sample=}\StringTok{ }\KeywordTok{subset}\NormalTok{ (autos[idx_sample,], }\DataTypeTok{select =}\NormalTok{ columns_numeric) }
\KeywordTok{pairs}\NormalTok{(autos_sample,}\DataTypeTok{col=}\StringTok{"dodgerblue"}\NormalTok{)}
\end{Highlighting}
\end{Shaded}

\includegraphics{team-a-final-proj_files/figure-latex/unnamed-chunk-5-1.pdf}

\begin{Shaded}
\begin{Highlighting}[]
\KeywordTok{round}\NormalTok{(}\KeywordTok{cor}\NormalTok{(autos_sample), }\DecValTok{2}\NormalTok{)}
\end{Highlighting}
\end{Shaded}

\begin{verbatim}
##                    price powerPS yearOfRegistration kilometer
## price               1.00    0.33               0.59     -0.45
## powerPS             0.33    1.00               0.08      0.01
## yearOfRegistration  0.59    0.08               1.00     -0.62
## kilometer          -0.45    0.01              -0.62      1.00
\end{verbatim}

Based on our observation, we have determined there little to no
collineararity issues wiht our model

\hypertarget{setup-training-and-testing-data}{%
\subsubsection{2.0.3 Setup training and testing
data}\label{setup-training-and-testing-data}}

\begin{Shaded}
\begin{Highlighting}[]
\CommentTok{#Setup data with 10000 randomly sampled and all columns}
\NormalTok{training_size=}\DecValTok{50000}
\KeywordTok{set.seed}\NormalTok{(}\DecValTok{20200807}\NormalTok{)}
\NormalTok{idx_train=}\KeywordTok{sample}\NormalTok{(}\DecValTok{1}\OperatorTok{:}\KeywordTok{nrow}\NormalTok{(autos),training_size)}
\NormalTok{autos_train=}\StringTok{ }\NormalTok{autos[idx_train,]}
\NormalTok{autos_train[,columns_factor]=}\KeywordTok{lapply}\NormalTok{(autos_train[,columns_factor], as.factor)}
\KeywordTok{str}\NormalTok{(autos_train)}
\end{Highlighting}
\end{Shaded}

\begin{verbatim}
## Classes 'tbl_df', 'tbl' and 'data.frame':    50000 obs. of  9 variables:
##  $ price             : num  1600 11950 4399 10900 3999 ...
##  $ abtest            : Factor w/ 2 levels "control","test": 2 2 2 1 1 1 1 1 2 1 ...
##  $ vehicleType       : Factor w/ 8 levels "andere","bus",..: 2 7 7 7 7 5 6 6 5 6 ...
##  $ yearOfRegistration: num  2004 2009 2003 2009 2008 ...
##  $ gearbox           : Factor w/ 2 levels "automatik","manuell": 2 1 2 2 2 2 2 2 2 1 ...
##  $ powerPS           : num  75 177 140 170 75 101 170 270 103 184 ...
##  $ kilometer         : num  150000 150000 150000 150000 70000 150000 150000 150000 90000 30000 ...
##  $ fuelType          : Factor w/ 7 levels "andere","benzin",..: 4 4 4 4 2 4 4 4 2 4 ...
##  $ notRepairedDamage : Factor w/ 2 levels "ja","nein": 2 2 2 2 2 2 2 2 2 2 ...
##  - attr(*, "na.action")= 'omit' Named int  1 2 6 9 16 18 21 25 26 28 ...
##   ..- attr(*, "names")= chr  "1" "2" "6" "9" ...
\end{verbatim}

\begin{Shaded}
\begin{Highlighting}[]
\CommentTok{#test data}
\NormalTok{test_size=training_size}
\NormalTok{idx_remain =}\StringTok{ }\OperatorTok{!}\NormalTok{(}\DecValTok{1}\OperatorTok{:}\KeywordTok{nrow}\NormalTok{(autos) }\OperatorTok\StringTok{ }\NormalTok{idx_train)}
\NormalTok{autos_remain =}\StringTok{ }\NormalTok{autos[idx_remain, ]}
\NormalTok{idx_test=}\KeywordTok{sample}\NormalTok{(}\DecValTok{1}\OperatorTok{:}\KeywordTok{nrow}\NormalTok{(autos_remain),test_size)}
\NormalTok{autos_test=}\StringTok{ }\NormalTok{autos_remain[idx_test,]}
\NormalTok{autos_test[,columns_factor]=}\KeywordTok{lapply}\NormalTok{(autos_test[,columns_factor], as.factor)}
\KeywordTok{rm}\NormalTok{(autos_remain)}
\end{Highlighting}
\end{Shaded}

\hypertarget{method-1---a-straightforward-but-dump-approach}{%
\subsection{2.1 Method 1 - a straightforward but dump
approach}\label{method-1---a-straightforward-but-dump-approach}}

\begin{Shaded}
\begin{Highlighting}[]
\NormalTok{model1_start=}\KeywordTok{lm}\NormalTok{(price}\OperatorTok{~}\NormalTok{.,}\DataTypeTok{data=}\NormalTok{autos_train)}
\CommentTok{#size of staring model}
\KeywordTok{length}\NormalTok{( }\KeywordTok{coef}\NormalTok{(model1_start) )}
\end{Highlighting}
\end{Shaded}

\begin{verbatim}
## [1] 20
\end{verbatim}

\begin{Shaded}
\begin{Highlighting}[]
\CommentTok{#Run BIC backward search}
\NormalTok{n=}\KeywordTok{nrow}\NormalTok{(autos_train)}
\NormalTok{model1_bic =}\StringTok{ }\KeywordTok{step}\NormalTok{(model1_start,}\DataTypeTok{k=}\KeywordTok{log}\NormalTok{(n),}\DataTypeTok{trace =} \DecValTok{0}\NormalTok{)}
\CommentTok{#size of selected model by backward BIC}
\KeywordTok{length}\NormalTok{( }\KeywordTok{coef}\NormalTok{(model1_bic) )}
\end{Highlighting}
\end{Shaded}

\begin{verbatim}
## [1] 19
\end{verbatim}

\begin{Shaded}
\begin{Highlighting}[]
\NormalTok{model1_bic}
\end{Highlighting}
\end{Shaded}

\begin{verbatim}
## 
## Call:
## lm(formula = price ~ vehicleType + yearOfRegistration + gearbox + 
##     powerPS + kilometer + fuelType + notRepairedDamage, data = autos_train)
## 
## Coefficients:
##           (Intercept)         vehicleTypebus      vehicleTypecabrio  
##             -1.89e+06               5.07e+02               4.33e+03  
##      vehicleTypecoupe  vehicleTypekleinwagen       vehicleTypekombi  
##              5.85e+03              -1.94e+03               4.36e+02  
##  vehicleTypelimousine         vehicleTypesuv     yearOfRegistration  
##              1.06e+03               3.89e+03               9.48e+02  
##        gearboxmanuell                powerPS              kilometer  
##             -3.67e+03               8.69e+00              -3.83e-02  
##        fuelTypebenzin            fuelTypecng         fuelTypediesel  
##             -2.33e+02              -1.46e+03               8.43e+02  
##       fuelTypeelektro         fuelTypehybrid            fuelTypelpg  
##             -8.12e+03              -1.04e+03              -8.99e+02  
## notRepairedDamagenein  
##              1.98e+03
\end{verbatim}

\begin{Shaded}
\begin{Highlighting}[]
\CommentTok{#checking model assumption}
\KeywordTok{diagnostics}\NormalTok{(model1_bic)}
\end{Highlighting}
\end{Shaded}

\includegraphics{team-a-final-proj_files/figure-latex/unnamed-chunk-8-1.pdf}

\begin{Shaded}
\begin{Highlighting}[]
\NormalTok{model1_selected=model1_bic}

\KeywordTok{remove_high_influential_points_and_refit_model}\NormalTok{(model1_selected,autos_train)}
\end{Highlighting}
\end{Shaded}

\includegraphics{team-a-final-proj_files/figure-latex/unnamed-chunk-8-2.pdf}

\begin{verbatim}
## $removed.n
## [1] 1517
## 
## $removed.fraction
## [1] 0.03034
## 
## $new.model
## 
## Call:
## lm(formula = formula, data = data1, subset = !high_infl)
## 
## Coefficients:
##           (Intercept)         vehicleTypebus      vehicleTypecabrio  
##             -1.59e+06               1.84e+02               2.13e+03  
##      vehicleTypecoupe  vehicleTypekleinwagen       vehicleTypekombi  
##              1.63e+03              -3.37e+02              -3.71e+02  
##  vehicleTypelimousine         vehicleTypesuv     yearOfRegistration  
##              2.37e+02               1.66e+03               7.97e+02  
##        gearboxmanuell                powerPS              kilometer  
##             -1.11e+03               4.54e+01              -3.35e-02  
##           fuelTypecng         fuelTypediesel        fuelTypeelektro  
##             -3.73e+02               1.30e+03              -1.41e+03  
##        fuelTypehybrid            fuelTypelpg  notRepairedDamagenein  
##              3.73e+03              -8.57e+02               1.78e+03
\end{verbatim}

for the following method we pick a data sample\_size of 50000 and with
fit into an additive model with all of the variables to display the
lenght of coef then we ran backward BIC to remove fsome of the
predictors then we have a source where we place our functions to pull
our model assumption to find out if our model is credible. Based on the
results, we have concluded this is not a good model. Fitted Residuals
and Q-Q Plots indicated some sort variable tranformations are needed.

\hypertarget{method-2---with-variable-transformations}{%
\subsection{2.2 Method 2 - With variable
transformations}\label{method-2---with-variable-transformations}}

We will take a 2-phases approach to finding a good model:

\begin{itemize}
\tightlist
\item
  Phase I - the Continuous variables phase
\item
  Pahse II - Adding categorical variables the
\end{itemize}

\hypertarget{phase-1-working-with-continous-variable-to-determine-the-best-model-form-for-continous-columns}{%
\subsubsection{2.2.1 - Phase 1: working with continous variable to
determine the best model form for continous
columns}\label{phase-1-working-with-continous-variable-to-determine-the-best-model-form-for-continous-columns}}

\hypertarget{break-data-set-into-subgroups-by-fixing-factor-variable-values}{%
\paragraph{2.2.1.1 - break data-set into subgroups by fixing factor
variable
values}\label{break-data-set-into-subgroups-by-fixing-factor-variable-values}}

\begin{Shaded}
\begin{Highlighting}[]
\CommentTok{#listing out all factor columns}
\NormalTok{columns_factor}
\end{Highlighting}
\end{Shaded}

\begin{verbatim}
## [1] "abtest"            "vehicleType"       "gearbox"          
## [4] "fuelType"          "notRepairedDamage"
\end{verbatim}

\begin{Shaded}
\begin{Highlighting}[]
\NormalTok{autos_factor_groups=autos }\OperatorTok\StringTok{ }\KeywordTok{count}\NormalTok{ (abtest,vehicleType,gearbox,fuelType,notRepairedDamage)}
\NormalTok{autos_factor_groups=autos_factor_groups[}\KeywordTok{order}\NormalTok{(autos_factor_groups}\OperatorTok{$}\NormalTok{n,}\DataTypeTok{decreasing =} \OtherTok{TRUE}\NormalTok{),]}

\CommentTok{#structure of autos_factor_groups}
\KeywordTok{head}\NormalTok{(autos_factor_groups)}
\end{Highlighting}
\end{Shaded}

\begin{verbatim}
## # A tibble: 6 x 6
##   abtest  vehicleType gearbox fuelType notRepairedDamage     n
##   <chr>   <chr>       <chr>   <chr>    <chr>             <int>
## 1 test    kleinwagen  manuell benzin   nein              14560
## 2 control kleinwagen  manuell benzin   nein              13311
## 3 test    limousine   manuell benzin   nein              10072
## 4 control limousine   manuell benzin   nein               9423
## 5 test    kombi       manuell diesel   nein               7891
## 6 control kombi       manuell diesel   nein               7290
\end{verbatim}

\begin{Shaded}
\begin{Highlighting}[]
\CommentTok{#Total number of groups}
\KeywordTok{nrow}\NormalTok{(autos_factor_groups)}
\end{Highlighting}
\end{Shaded}

\begin{verbatim}
## [1] 269
\end{verbatim}

\begin{Shaded}
\begin{Highlighting}[]
\CommentTok{#Number of groups with more than 300 records}
\KeywordTok{sum}\NormalTok{(autos_factor_groups}\OperatorTok{$}\NormalTok{n}\OperatorTok{>}\DecValTok{300}\NormalTok{)}
\end{Highlighting}
\end{Shaded}

\begin{verbatim}
## [1] 66
\end{verbatim}

Our data exploration continued though a process of breaking our dataset
into subgroups by categorizing with the following variables: abtest,
vehicleType, gearbox, model, fuelType and brand. For example abtest can
only have either be test or control, and vehicleType can either be
Limousine or Kleinwagon, we have picked a number of groups with more
than 300 records in order to find a definitive model for our dataset.

\hypertarget{work-throght-one-subgroup-first.-choose-group-2}{%
\paragraph{2.2.1.2 - work throght one subgroup first. (choose group
2)}\label{work-throght-one-subgroup-first.-choose-group-2}}

\begin{Shaded}
\begin{Highlighting}[]
\CommentTok{#choose group}
\NormalTok{selected_group_idx=}\DecValTok{2}
\NormalTok{group1=autos_factor_groups[selected_group_idx, ]}
\NormalTok{(}\DataTypeTok{group1.size =}\NormalTok{ group1}\OperatorTok{$}\NormalTok{n)}
\end{Highlighting}
\end{Shaded}

\begin{verbatim}
## [1] 13311
\end{verbatim}

\begin{Shaded}
\begin{Highlighting}[]
\NormalTok{group1=}\KeywordTok{subset}\NormalTok{(group1, }\DataTypeTok{select =} \OperatorTok{-}\KeywordTok{c}\NormalTok{(n) )}

\CommentTok{#get the records for the selected group}
\NormalTok{autos_}\DecValTok{1}\NormalTok{=autos}
\NormalTok{cols=}\KeywordTok{names}\NormalTok{(group1)}
\ControlFlowTok{for}\NormalTok{ (i }\ControlFlowTok{in} \DecValTok{1}\OperatorTok{:}\KeywordTok{ncol}\NormalTok{(group1))\{}
\NormalTok{  idx =}\StringTok{ }\NormalTok{autos_}\DecValTok{1}\NormalTok{[,cols[i]]}\OperatorTok{==}\NormalTok{group1[[i]]}
\NormalTok{  autos_}\DecValTok{1}\NormalTok{=autos_}\DecValTok{1}\NormalTok{[idx,]}
\NormalTok{\}}
\NormalTok{autos_}\DecValTok{1}\NormalTok{=}\KeywordTok{subset}\NormalTok{(autos_}\DecValTok{1}\NormalTok{, }\DataTypeTok{select =}\NormalTok{ columns_numeric )}
\end{Highlighting}
\end{Shaded}

todo: pairs(autos\_1,col=``dodgerblue''), and explain the reason we
isolated the group

We will use the newly isolated dataset, \textbf{autos\_1} to better help
finding a model form for continuous variables

\hypertarget{try-an-additive-model-using-all-continous-variables}{%
\paragraph{2.2.1.3 - Try an additive model using all continous
variables}\label{try-an-additive-model-using-all-continous-variables}}

\begin{Shaded}
\begin{Highlighting}[]
\KeywordTok{par}\NormalTok{(}\DataTypeTok{mfrow=}\KeywordTok{c}\NormalTok{(}\DecValTok{1}\NormalTok{,}\DecValTok{2}\NormalTok{))}
\NormalTok{model2_add =}\StringTok{ }\KeywordTok{lm}\NormalTok{(price }\OperatorTok{~}\StringTok{ }\NormalTok{powerPS }\OperatorTok{+}\NormalTok{yearOfRegistration}\OperatorTok{+}\NormalTok{kilometer, }\DataTypeTok{data =}\NormalTok{ autos_}\DecValTok{1}\NormalTok{)}
\KeywordTok{diagnostics}\NormalTok{(model2_add)}
\end{Highlighting}
\end{Shaded}

\includegraphics{team-a-final-proj_files/figure-latex/unnamed-chunk-11-1.pdf}

The plots and test pValue indicates signifant voiloation of equal
variance and normality assumption. We will try Box-Cox transformation on
Price next.

\hypertarget{try-box-cox-tranformation-on-the-selected-group}{%
\paragraph{2.2.1.4 - Try Box-Cox tranformation on the selected
group}\label{try-box-cox-tranformation-on-the-selected-group}}

\begin{Shaded}
\begin{Highlighting}[]
\KeywordTok{par}\NormalTok{(}\DataTypeTok{mfrow=}\KeywordTok{c}\NormalTok{(}\DecValTok{1}\NormalTok{,}\DecValTok{1}\NormalTok{))}
\NormalTok{boxcox_input_formula=}\StringTok{ }\KeywordTok{as.formula}\NormalTok{ ( }\KeywordTok{as.character}\NormalTok{(model2_add}\OperatorTok{$}\NormalTok{call[}\DecValTok{2}\NormalTok{]) )}
\NormalTok{out=}\KeywordTok{boxcox}\NormalTok{(model2_add, }\DataTypeTok{plotit =} \OtherTok{TRUE}\NormalTok{, }\DataTypeTok{lambda =} \KeywordTok{seq}\NormalTok{(}\OperatorTok{-}\FloatTok{0.5}\NormalTok{, }\FloatTok{1.0}\NormalTok{, }\DataTypeTok{by =} \FloatTok{0.05}\NormalTok{))}
\end{Highlighting}
\end{Shaded}

\includegraphics{team-a-final-proj_files/figure-latex/unnamed-chunk-12-1.pdf}

\begin{Shaded}
\begin{Highlighting}[]
\NormalTok{( }\DataTypeTok{lambda=}\NormalTok{out}\OperatorTok{$}\NormalTok{x[}\KeywordTok{which.max}\NormalTok{(out}\OperatorTok{$}\NormalTok{y)] )}
\end{Highlighting}
\end{Shaded}

\begin{verbatim}
## [1] 0.1212
\end{verbatim}

\begin{Shaded}
\begin{Highlighting}[]
\NormalTok{model2_add_cox =}\StringTok{ }\KeywordTok{lm}\NormalTok{( ((price}\OperatorTok{^}\NormalTok{lambda}\DecValTok{-1}\NormalTok{)}\OperatorTok{/}\NormalTok{lambda) }\OperatorTok{~}\StringTok{ }\NormalTok{powerPS }\OperatorTok{+}\NormalTok{yearOfRegistration , }\DataTypeTok{data =}\NormalTok{ autos_}\DecValTok{1}\NormalTok{ )}
\KeywordTok{diagnostics}\NormalTok{(model2_add_cox)}
\end{Highlighting}
\end{Shaded}

\includegraphics{team-a-final-proj_files/figure-latex/unnamed-chunk-12-2.pdf}

Plots looks better but it seems additional predictor transformation
might help.

\hypertarget{finding-predictor-tranformation-via-backward-bic-search}{%
\paragraph{2.2.1.5 - finding predictor tranformation via backward BIC
search}\label{finding-predictor-tranformation-via-backward-bic-search}}

\begin{Shaded}
\begin{Highlighting}[]
\CommentTok{# starting with log and 2nd order terms for all predictors}
\NormalTok{model2a_bic_start=}\KeywordTok{lm}\NormalTok{( (price}\OperatorTok{^}\NormalTok{lambda}\DecValTok{-1}\NormalTok{)}\OperatorTok{/}\NormalTok{lambda }\OperatorTok{~}\StringTok{ }\NormalTok{powerPS}\OperatorTok{+}\KeywordTok{I}\NormalTok{(}\KeywordTok{log}\NormalTok{(powerPS)) }\OperatorTok{+}\StringTok{ }\KeywordTok{I}\NormalTok{(powerPS}\OperatorTok{^}\DecValTok{2}\NormalTok{)}
                        \OperatorTok{+}\NormalTok{yearOfRegistration }\OperatorTok{+}\StringTok{ }\KeywordTok{I}\NormalTok{(}\KeywordTok{log}\NormalTok{(yearOfRegistration)) }\OperatorTok{+}\StringTok{ }\KeywordTok{I}\NormalTok{(yearOfRegistration}\OperatorTok{^}\DecValTok{2}\NormalTok{) }
                        \OperatorTok{+}\NormalTok{kilometer }\OperatorTok{+}\StringTok{ }\KeywordTok{I}\NormalTok{(}\KeywordTok{log}\NormalTok{(kilometer)) }\OperatorTok{+}\StringTok{ }\KeywordTok{I}\NormalTok{(kilometer}\OperatorTok{^}\DecValTok{2}\NormalTok{)}
\NormalTok{                      , }\DataTypeTok{data =}\NormalTok{ autos_}\DecValTok{1}\NormalTok{ )}
  



\NormalTok{model2a_bic =}\StringTok{ }\KeywordTok{step}\NormalTok{(model2a_bic_start,}\DataTypeTok{trace=}\DecValTok{0}\NormalTok{)}
\KeywordTok{diagnostics}\NormalTok{(model2a_bic)}
\end{Highlighting}
\end{Shaded}

\includegraphics{team-a-final-proj_files/figure-latex/unnamed-chunk-13-1.pdf}

\begin{Shaded}
\begin{Highlighting}[]
\KeywordTok{summary}\NormalTok{(model2a_bic)}
\end{Highlighting}
\end{Shaded}

\begin{verbatim}
## 
## Call:
## lm(formula = (price^lambda - 1)/lambda ~ powerPS + I(log(powerPS)) + 
##     I(powerPS^2) + I(log(yearOfRegistration)) + kilometer + I(log(kilometer)), 
##     data = autos_1)
## 
## Residuals:
##    Min     1Q Median     3Q    Max 
## -5.293 -0.528  0.009  0.562  6.700 
## 
## Coefficients:
##                             Estimate Std. Error t value Pr(>|t|)    
## (Intercept)                -4.58e+03   4.22e+01 -108.55  < 2e-16 ***
## powerPS                    -3.32e-03   3.61e-04   -9.20  < 2e-16 ***
## I(log(powerPS))             2.62e+00   4.73e-02   55.36  < 2e-16 ***
## I(powerPS^2)                2.10e-07   4.41e-08    4.76  1.9e-06 ***
## I(log(yearOfRegistration))  6.03e+02   5.55e+00  108.51  < 2e-16 ***
## kilometer                  -1.50e-05   5.03e-07  -29.83  < 2e-16 ***
## I(log(kilometer))           3.79e-01   3.31e-02   11.48  < 2e-16 ***
## ---
## Signif. codes:  0 '***' 0.001 '**' 0.01 '*' 0.05 '.' 0.1 ' ' 1
## 
## Residual standard error: 0.865 on 13304 degrees of freedom
## Multiple R-squared:  0.823,  Adjusted R-squared:  0.823 
## F-statistic: 1.03e+04 on 6 and 13304 DF,  p-value: <2e-16
\end{verbatim}

\begin{Shaded}
\begin{Highlighting}[]
\CommentTok{#Save model2a for later steps}
\NormalTok{model2a=model2a_bic}
\NormalTok{formula_str=}\KeywordTok{as.character}\NormalTok{(model2a}\OperatorTok{$}\NormalTok{call[}\DecValTok{2}\NormalTok{])}
\end{Highlighting}
\end{Shaded}

The formula form of model(using only continuous variables) is:

\begin{itemize}
\tightlist
\item
  \textbf{(price\^{}lambda - 1)/lambda \textasciitilde{} powerPS +
  I(log(powerPS)) + I(powerPS\^{}2) + I(log(yearOfRegistration)) +
  kilometer + I(log(kilometer))}
\end{itemize}

\hypertarget{to-be-removed--run-box-cox-tranformation-all-for-groups-with-300-records}{%
\paragraph{2.2.1.6 (to be removed)- Run box-cox tranformation all for
groups with \textgreater{} 300
records}\label{to-be-removed--run-box-cox-tranformation-all-for-groups-with-300-records}}

We will run box-cox transformation on subgroup of data that has more
than 300 records..

\begin{Shaded}
\begin{Highlighting}[]
\KeywordTok{source}\NormalTok{(}\StringTok{"misc_functions.R"}\NormalTok{)}
\NormalTok{boxcox_input_formula=}\StringTok{ }\KeywordTok{as.formula}\NormalTok{ ( }\KeywordTok{as.character}\NormalTok{(model2_add}\OperatorTok{$}\NormalTok{call[}\DecValTok{2}\NormalTok{]) )}
\NormalTok{(}\DataTypeTok{boxcox_lambda=}\KeywordTok{subset_autodata_with_boxcox}\NormalTok{(autos,boxcox_input_formula))}
\KeywordTok{hist}\NormalTok{(boxcox_lambda,}\DataTypeTok{breaks=}\DecValTok{20}\NormalTok{,}\DataTypeTok{col=}\StringTok{"lightblue"}\NormalTok{)}
\NormalTok{lambda=}\KeywordTok{mean}\NormalTok{(boxcox_lambda)}
\NormalTok{lambda}
\end{Highlighting}
\end{Shaded}

Based on our analysis above, we will use \(\lambda=0.1212\) for the
Box-Cox transformation!

\hypertarget{phase-2-adding-categorical-variables-to-the-model-that-was-determined-in-phase-1}{%
\subsubsection{2.2.2 - Phase 2: adding categorical variables to the
model that was determined in Phase
1}\label{phase-2-adding-categorical-variables-to-the-model-that-was-determined-in-phase-1}}

The best model form with only continous variable is:

\begin{itemize}
\tightlist
\item
  (price\^{}lambda - 1)/lambda \textasciitilde{} powerPS +
  I(log(powerPS)) + I(powerPS\^{}2) + I(log(yearOfRegistration)) +
  kilometer + I(log(kilometer))\\
\item
  Where we determine the new \(\lambda\) value based on entire training
  dataset
\end{itemize}

Next, we will add all factor varibles to the formula format we obtained
from the Phase 1 as the starting model in backward BIC search.

\hypertarget{new---run-box-cox-tranformation-on-all-training-data}{%
\paragraph{2.2.2.0 (new) - Run box-cox tranformation on all training
data}\label{new---run-box-cox-tranformation-on-all-training-data}}

We will determine new \(\lambda\) value by fitting established model
form on all training data

\begin{itemize}
\tightlist
\item
  \textbf{Formula format determined from Phase 1}: (price\^{}lambda -
  1)/lambda \textasciitilde{} powerPS + I(log(powerPS)) +
  I(powerPS\^{}2) + I(log(yearOfRegistration)) + kilometer +
  I(log(kilometer))
\end{itemize}

\begin{Shaded}
\begin{Highlighting}[]
\KeywordTok{par}\NormalTok{(}\DataTypeTok{mfrow=}\KeywordTok{c}\NormalTok{(}\DecValTok{1}\NormalTok{,}\DecValTok{1}\NormalTok{))}

\NormalTok{model_2a_all=}\StringTok{ }\KeywordTok{lm}\NormalTok{ (price }\OperatorTok{~}\StringTok{ }\NormalTok{powerPS }\OperatorTok{+}\StringTok{ }\KeywordTok{I}\NormalTok{(}\KeywordTok{log}\NormalTok{(powerPS)) }\OperatorTok{+}\StringTok{ }\KeywordTok{I}\NormalTok{(powerPS}\OperatorTok{^}\DecValTok{2}\NormalTok{) }\OperatorTok{+}\StringTok{ }\KeywordTok{I}\NormalTok{(}\KeywordTok{log}\NormalTok{(yearOfRegistration)) }\OperatorTok{+}\StringTok{ }\NormalTok{kilometer }\OperatorTok{+}\StringTok{ }\KeywordTok{I}\NormalTok{(}\KeywordTok{log}\NormalTok{(kilometer)),}\DataTypeTok{data=}\NormalTok{autos_train)}

\KeywordTok{diagnostics}\NormalTok{(model_2a_all)}
\end{Highlighting}
\end{Shaded}

\includegraphics{team-a-final-proj_files/figure-latex/unnamed-chunk-15-1.pdf}

\begin{Shaded}
\begin{Highlighting}[]
\NormalTok{out=}\KeywordTok{boxcox}\NormalTok{(model_2a_all, }\DataTypeTok{plotit =} \OtherTok{TRUE}\NormalTok{, }\DataTypeTok{lambda =} \KeywordTok{seq}\NormalTok{(}\OperatorTok{-}\FloatTok{0.5}\NormalTok{, }\FloatTok{1.0}\NormalTok{, }\DataTypeTok{by =} \FloatTok{0.05}\NormalTok{))}
\NormalTok{( }\DataTypeTok{lambda=}\NormalTok{out}\OperatorTok{$}\NormalTok{x[}\KeywordTok{which.max}\NormalTok{(out}\OperatorTok{$}\NormalTok{y)] )}
\end{Highlighting}
\end{Shaded}

\begin{verbatim}
## [1] 0.1212
\end{verbatim}

\includegraphics{team-a-final-proj_files/figure-latex/unnamed-chunk-15-2.pdf}

\hypertarget{searching-for-a-good-model-using-backward-bic}{%
\paragraph{2.2.2.1 Searching for a good model using backward
BIC}\label{searching-for-a-good-model-using-backward-bic}}

\begin{Shaded}
\begin{Highlighting}[]
\CommentTok{#model2_start = lm( ((price^lambda - 1)/lambda) ~ .-vehicleType +  I(powerPS^2), data=autos_train ) }
\NormalTok{model2_start =}\StringTok{ }\KeywordTok{lm}\NormalTok{( ((price}\OperatorTok{^}\NormalTok{lambda }\OperatorTok{-}\StringTok{ }\DecValTok{1}\NormalTok{)}\OperatorTok{/}\NormalTok{lambda) }\OperatorTok{~}\StringTok{ }\NormalTok{.}\OperatorTok{+}\KeywordTok{I}\NormalTok{(}\KeywordTok{log}\NormalTok{(powerPS)) }\OperatorTok{+}\StringTok{ }\KeywordTok{I}\NormalTok{(powerPS}\OperatorTok{^}\DecValTok{2}\NormalTok{) }\OperatorTok{+}\StringTok{ }\KeywordTok{I}\NormalTok{(}\KeywordTok{log}\NormalTok{(yearOfRegistration)) }\OperatorTok{+}\StringTok{ }\KeywordTok{I}\NormalTok{(}\KeywordTok{log}\NormalTok{(kilometer)), }\DataTypeTok{data=}\NormalTok{autos_train ) }

\NormalTok{n=}\KeywordTok{nrow}\NormalTok{(autos_train)}
\NormalTok{model2_selected_bic =}\StringTok{ }\KeywordTok{step}\NormalTok{(model2_start,}\DataTypeTok{k=}\KeywordTok{log}\NormalTok{(n),}\DataTypeTok{trace=}\DecValTok{0}\NormalTok{)}
\KeywordTok{summary}\NormalTok{(model2_selected_bic)}
\end{Highlighting}
\end{Shaded}

\begin{verbatim}
## 
## Call:
## lm(formula = ((price^lambda - 1)/lambda) ~ vehicleType + gearbox + 
##     powerPS + kilometer + fuelType + notRepairedDamage + I(log(powerPS)) + 
##     I(log(yearOfRegistration)) + I(log(kilometer)), data = autos_train)
## 
## Residuals:
##     Min      1Q  Median      3Q     Max 
## -12.205  -0.638   0.044   0.678  14.248 
## 
## Coefficients:
##                             Estimate Std. Error t value Pr(>|t|)    
## (Intercept)                -5.48e+03   2.70e+01 -202.79  < 2e-16 ***
## vehicleTypebus              8.50e-02   7.04e-02    1.21  0.22736    
## vehicleTypecabrio           9.07e-01   7.17e-02   12.66  < 2e-16 ***
## vehicleTypecoupe            5.28e-01   7.29e-02    7.25  4.3e-13 ***
## vehicleTypekleinwagen      -2.56e-02   7.00e-02   -0.36  0.71517    
## vehicleTypekombi           -2.48e-01   6.97e-02   -3.56  0.00037 ***
## vehicleTypelimousine        5.94e-02   6.96e-02    0.85  0.39293    
## vehicleTypesuv              4.82e-01   7.23e-02    6.67  2.7e-11 ***
## gearboxmanuell             -3.62e-01   1.27e-02  -28.56  < 2e-16 ***
## powerPS                    -1.15e-03   4.24e-05  -27.08  < 2e-16 ***
## kilometer                  -1.60e-05   3.74e-07  -42.84  < 2e-16 ***
## fuelTypebenzin             -1.16e+00   6.42e-01   -1.81  0.07004 .  
## fuelTypecng                -1.23e+00   6.51e-01   -1.89  0.05850 .  
## fuelTypediesel             -7.81e-01   6.42e-01   -1.22  0.22388    
## fuelTypeelektro             1.06e+00   7.54e-01    1.41  0.15832    
## fuelTypehybrid             -5.90e-01   6.58e-01   -0.90  0.37061    
## fuelTypelpg                -1.33e+00   6.44e-01   -2.06  0.03921 *  
## notRepairedDamagenein       1.31e+00   1.93e-02   67.67  < 2e-16 ***
## I(log(powerPS))             2.85e+00   1.87e-02  152.50  < 2e-16 ***
## I(log(yearOfRegistration))  7.20e+02   3.55e+00  202.63  < 2e-16 ***
## I(log(kilometer))           5.50e-01   2.49e-02   22.14  < 2e-16 ***
## ---
## Signif. codes:  0 '***' 0.001 '**' 0.01 '*' 0.05 '.' 0.1 ' ' 1
## 
## Residual standard error: 1.11 on 49979 degrees of freedom
## Multiple R-squared:  0.825,  Adjusted R-squared:  0.825 
## F-statistic: 1.18e+04 on 20 and 49979 DF,  p-value: <2e-16
\end{verbatim}

\begin{Shaded}
\begin{Highlighting}[]
\KeywordTok{diagnostics}\NormalTok{(model2_selected_bic)}
\end{Highlighting}
\end{Shaded}

\includegraphics{team-a-final-proj_files/figure-latex/unnamed-chunk-16-1.pdf}

\begin{Shaded}
\begin{Highlighting}[]
\NormalTok{model2_selected=model2_selected_bic}

\NormalTok{model2_selected_bic}\OperatorTok{$}\NormalTok{call[}\DecValTok{2}\NormalTok{]}
\end{Highlighting}
\end{Shaded}

\begin{verbatim}
## (((price^lambda - 1)/lambda) ~ vehicleType + gearbox + powerPS + 
##     kilometer + fuelType + notRepairedDamage + I(log(powerPS)) + 
##     I(log(yearOfRegistration)) + I(log(kilometer)))()
\end{verbatim}

\hypertarget{a-verify-regression-signifcant-with-anova-test-on-high-p-value-betasvehicletype-and-fueltype}{%
\paragraph{2.2.2.1a verify regression signifcant with ANOVA test on high
p-Value betas(vehicleType and
fuelType)}\label{a-verify-regression-signifcant-with-anova-test-on-high-p-value-betasvehicletype-and-fueltype}}

-vehicleType

\begin{Shaded}
\begin{Highlighting}[]
\NormalTok{null_model_str =}\StringTok{ }\KeywordTok{paste}\NormalTok{ (}\KeywordTok{as.character}\NormalTok{(model2_selected_bic}\OperatorTok{$}\NormalTok{call[}\DecValTok{2}\NormalTok{]), }\StringTok{"-vehicleType"}\NormalTok{)}
\NormalTok{model_null=}\StringTok{ }\KeywordTok{lm}\NormalTok{(null_model_str, }\DataTypeTok{data =}\NormalTok{ autos_train)}
\KeywordTok{anova}\NormalTok{(model_null,model2_selected_bic)}
\end{Highlighting}
\end{Shaded}

\begin{verbatim}
## Analysis of Variance Table
## 
## Model 1: ((price^lambda - 1)/lambda) ~ vehicleType + gearbox + powerPS + 
##     kilometer + fuelType + notRepairedDamage + I(log(powerPS)) + 
##     I(log(yearOfRegistration)) + I(log(kilometer)) - vehicleType
## Model 2: ((price^lambda - 1)/lambda) ~ vehicleType + gearbox + powerPS + 
##     kilometer + fuelType + notRepairedDamage + I(log(powerPS)) + 
##     I(log(yearOfRegistration)) + I(log(kilometer))
##   Res.Df   RSS Df Sum of Sq   F Pr(>F)    
## 1  49986 66008                            
## 2  49979 61852  7      4156 480 <2e-16 ***
## ---
## Signif. codes:  0 '***' 0.001 '**' 0.01 '*' 0.05 '.' 0.1 ' ' 1
\end{verbatim}

-fuelType

\begin{Shaded}
\begin{Highlighting}[]
\NormalTok{null_model_str =}\StringTok{ }\KeywordTok{paste}\NormalTok{ (}\KeywordTok{as.character}\NormalTok{(model2_selected_bic}\OperatorTok{$}\NormalTok{call[}\DecValTok{2}\NormalTok{]), }\StringTok{"-fuelType"}\NormalTok{)}
\NormalTok{model_null=}\StringTok{ }\KeywordTok{lm}\NormalTok{(null_model_str, }\DataTypeTok{data =}\NormalTok{ autos_train)}
\KeywordTok{anova}\NormalTok{(model_null,model2_selected_bic)}
\end{Highlighting}
\end{Shaded}

\begin{verbatim}
## Analysis of Variance Table
## 
## Model 1: ((price^lambda - 1)/lambda) ~ vehicleType + gearbox + powerPS + 
##     kilometer + fuelType + notRepairedDamage + I(log(powerPS)) + 
##     I(log(yearOfRegistration)) + I(log(kilometer)) - fuelType
## Model 2: ((price^lambda - 1)/lambda) ~ vehicleType + gearbox + powerPS + 
##     kilometer + fuelType + notRepairedDamage + I(log(powerPS)) + 
##     I(log(yearOfRegistration)) + I(log(kilometer))
##   Res.Df   RSS Df Sum of Sq   F Pr(>F)    
## 1  49985 63181                            
## 2  49979 61852  6      1329 179 <2e-16 ***
## ---
## Signif. codes:  0 '***' 0.001 '**' 0.01 '*' 0.05 '.' 0.1 ' ' 1
\end{verbatim}

\hypertarget{remove-high-influence-data-and-refit-the-phase-2-model}{%
\paragraph{2.2.2.2 Remove high influence data and refit the Phase 2
model}\label{remove-high-influence-data-and-refit-the-phase-2-model}}

\begin{Shaded}
\begin{Highlighting}[]
\NormalTok{fix_model=}\KeywordTok{remove_high_influential_points_and_refit_model}\NormalTok{(model2_selected_bic, autos_train)}
\end{Highlighting}
\end{Shaded}

\includegraphics{team-a-final-proj_files/figure-latex/unnamed-chunk-19-1.pdf}

\begin{Shaded}
\begin{Highlighting}[]
\NormalTok{fix_model}
\end{Highlighting}
\end{Shaded}

\begin{verbatim}
## $removed.n
## [1] 1965
## 
## $removed.fraction
## [1] 0.0393
## 
## $new.model
## 
## Call:
## lm(formula = formula, data = data1, subset = !high_infl)
## 
## Coefficients:
##                (Intercept)              vehicleTypebus  
##                  -5.58e+03                    6.96e-02  
##          vehicleTypecabrio            vehicleTypecoupe  
##                   8.23e-01                    4.25e-01  
##      vehicleTypekleinwagen            vehicleTypekombi  
##                  -1.31e-01                   -2.16e-01  
##       vehicleTypelimousine              vehicleTypesuv  
##                   8.29e-02                    4.43e-01  
##             gearboxmanuell                     powerPS  
##                  -3.32e-01                    6.05e-03  
##                  kilometer                 fuelTypecng  
##                  -1.60e-05                   -2.99e-02  
##             fuelTypediesel              fuelTypehybrid  
##                   4.05e-01                    5.43e-01  
##                fuelTypelpg       notRepairedDamagenein  
##                  -1.41e-01                    1.22e+00  
##            I(log(powerPS))  I(log(yearOfRegistration))  
##                   1.84e+00                    7.34e+02  
##          I(log(kilometer))  
##                   5.82e-01
\end{verbatim}

\hypertarget{method-3---find-an-interaction-model-based-on-method-2-optional}{%
\subsection{2.3 Method 3 - find an interaction model based on method 2
(optional)}\label{method-3---find-an-interaction-model-based-on-method-2-optional}}

We will do this if we have time

\begin{Shaded}
\begin{Highlighting}[]
\KeywordTok{library}\NormalTok{(tictoc)}
\NormalTok{lambda}
\NormalTok{model3a_bic_start=}\KeywordTok{lm}\NormalTok{( (price}\OperatorTok{^}\NormalTok{lambda}\DecValTok{-1}\NormalTok{)}\OperatorTok{/}\NormalTok{lambda }\OperatorTok{~}\StringTok{ }\NormalTok{.}\OperatorTok{^}\DecValTok{2}
                      \OperatorTok{+}\StringTok{ }\KeywordTok{I}\NormalTok{(}\KeywordTok{log}\NormalTok{(powerPS)) }\OperatorTok{+}\StringTok{ }\KeywordTok{I}\NormalTok{(powerPS}\OperatorTok{^}\DecValTok{2}\NormalTok{)}
                      \OperatorTok{+}\StringTok{ }\KeywordTok{I}\NormalTok{(}\KeywordTok{log}\NormalTok{(yearOfRegistration)) }\OperatorTok{+}\StringTok{ }\KeywordTok{I}\NormalTok{(yearOfRegistration}\OperatorTok{^}\DecValTok{2}\NormalTok{) }
                      \OperatorTok{+}\StringTok{ }\KeywordTok{I}\NormalTok{(}\KeywordTok{log}\NormalTok{(kilometer)) }\OperatorTok{+}\StringTok{ }\KeywordTok{I}\NormalTok{(kilometer}\OperatorTok{^}\DecValTok{2}\NormalTok{)}
\NormalTok{                      , }\DataTypeTok{data =}\NormalTok{ autos_}\DecValTok{1}\NormalTok{ )}

\NormalTok{model3a_bic =}\StringTok{ }\KeywordTok{step}\NormalTok{(model2a_bic_start,}\DataTypeTok{trace=}\DecValTok{0}\NormalTok{)}
\KeywordTok{diagnostics}\NormalTok{(model3a_bic)}
\KeywordTok{summary}\NormalTok{(model3a_bic)}
\end{Highlighting}
\end{Shaded}

\hypertarget{results}{%
\section{3. Results}\label{results}}

\begin{Shaded}
\begin{Highlighting}[]
\CommentTok{# function to evaluate rmse}
\NormalTok{rmse  =}\StringTok{ }\ControlFlowTok{function}\NormalTok{(actual, predicted) \{}
  \KeywordTok{sqrt}\NormalTok{(}\KeywordTok{mean}\NormalTok{((actual }\OperatorTok{-}\StringTok{ }\NormalTok{predicted) }\OperatorTok{^}\StringTok{ }\DecValTok{2}\NormalTok{))}
\NormalTok{\}}

\CommentTok{#mod1 RMSEs}
\NormalTok{Train_RMSE_mod1 =}\StringTok{ }\KeywordTok{rmse}\NormalTok{(autos_train}\OperatorTok{$}\NormalTok{price, }\KeywordTok{predict}\NormalTok{(model1_selected, autos_train, }\DataTypeTok{type =} \StringTok{'response'}\NormalTok{))}
\NormalTok{Test_RMSE_mod1 =}\StringTok{  }\KeywordTok{rmse}\NormalTok{(autos_test}\OperatorTok{$}\NormalTok{price, }\KeywordTok{predict}\NormalTok{(model1_selected, autos_test,}\DataTypeTok{type =} \StringTok{'response'}\NormalTok{))}

\CommentTok{#mod2 RMSEs}
\NormalTok{Train_RMSE_mod2 =}\StringTok{ }\KeywordTok{rmse}\NormalTok{(autos_train}\OperatorTok{$}\NormalTok{price, (}\KeywordTok{predict}\NormalTok{(model2_selected, }\DataTypeTok{newdata =}\NormalTok{ autos_train, }\DataTypeTok{type =}\StringTok{'response'}\NormalTok{)}\OperatorTok{*}\NormalTok{lambda}\OperatorTok{+}\DecValTok{1}\NormalTok{)}\OperatorTok{^}\NormalTok{(}\DecValTok{1}\OperatorTok{/}\NormalTok{lambda))}
\NormalTok{Test_RMSE_mod2 =}\StringTok{ }\KeywordTok{rmse}\NormalTok{(autos_test}\OperatorTok{$}\NormalTok{price, (}\KeywordTok{predict}\NormalTok{(model2_selected, }\DataTypeTok{newdata =}\NormalTok{ autos_test, }\DataTypeTok{type =}\StringTok{'response'}\NormalTok{)}\OperatorTok{*}\NormalTok{lambda}\OperatorTok{+}\DecValTok{1}\NormalTok{)}\OperatorTok{^}\NormalTok{(}\DecValTok{1}\OperatorTok{/}\NormalTok{lambda))}

\CommentTok{# calculate all train errors}
\NormalTok{train_error =}\StringTok{ }\KeywordTok{c}\NormalTok{(Train_RMSE_mod1, Train_RMSE_mod2)}

\CommentTok{# calculate all test errors}
\NormalTok{test_error =}\StringTok{ }\KeywordTok{c}\NormalTok{(Test_RMSE_mod1, Test_RMSE_mod2)}

\NormalTok{auto_models =}\StringTok{ }\KeywordTok{c}\NormalTok{(}\StringTok{"`Additive model`"}\NormalTok{, }\StringTok{"`Transformation model`"}\NormalTok{)}
\NormalTok{auto_results =}\StringTok{ }\KeywordTok{data.frame}\NormalTok{(auto_models, train_error, test_error)}
\KeywordTok{colnames}\NormalTok{(auto_results) =}\StringTok{ }\KeywordTok{c}\NormalTok{(}\StringTok{"Model"}\NormalTok{, }\StringTok{"Train RMSE"}\NormalTok{, }\StringTok{"Test RMSE"}\NormalTok{)}
\NormalTok{knitr}\OperatorTok{::}\KeywordTok{kable}\NormalTok{(auto_results)}
\end{Highlighting}
\end{Shaded}

\begin{longtable}[]{@{}lrr@{}}
\toprule
Model & Train RMSE & Test RMSE\tabularnewline
\midrule
\endhead
\texttt{Additive\ model} & 5883 & 5960\tabularnewline
\texttt{Transformation\ model} & 4345 & 4493\tabularnewline
\bottomrule
\end{longtable}

\hypertarget{discussion}{%
\section{4. Discussion}\label{discussion}}

\hypertarget{appendix}{%
\section{5. Appendix}\label{appendix}}

\hypertarget{the-listing-for-r-code-used}{%
\paragraph{The listing for R code
used}\label{the-listing-for-r-code-used}}

\begin{Shaded}
\begin{Highlighting}[]
\FunctionTok{cat}\NormalTok{ -n misc_functions.R}
\end{Highlighting}
\end{Shaded}

\begin{verbatim}
     1  
     2  get_boxcox_lambda = function(model1){
     3    boxcot_out=boxcox(model1, plotit = FALSE, lambda = seq(-0.5, 1.0, by = 0.05))
     4    lambda=boxcot_out$x[which.max(boxcot_out$y)]
     5    return(lambda)
     6  }
     7  
     8  subset_autodata_with_boxcox =function(data,input_formula){
     9    
    10    autos_factor_groups=data %>%count(abtest,vehicleType,gearbox,fuelType,notRepairedDamage)
    11    autos_factor_groups=autos_factor_groups[order(autos_factor_groups$n,decreasing = TRUE),]
    12    autos_factor_groups=autos_factor_groups[autos_factor_groups$n>300,]
    13    nGroup=nrow(autos_factor_groups)
    14    #nGroup=5
    15    lambda_bc=rep(0,nGroup)
    16    
    17    for (g in 1:nGroup){
    18      group1=autos_factor_groups[g,]
    19      group1.size = group1$n
    20      group1=subset(group1, select = -c(n) )
    21      
    22      autos_1=data
    23      cols=names(group1)
    24      for (i in 1:ncol(group1)){
    25        idx = autos_1[,cols[i]]==group1[[i]]
    26        autos_1=autos_1[idx,]
    27      }
    28      model = lm(input_formula, data = autos_1,y=TRUE, qr=TRUE)
    29      lambda_bc[g] = get_boxcox_lambda(model)
    30    }
    31    return (lambda_bc)
    32  }
    33  
    34  # source("misc_functions.R")
    35  remove_high_influential_points_and_refit_model = function (model, data1){
    36    ret = list()
    37    #finding influenctial
    38    cd = cooks.distance(model)
    39    n=length(resid(model))
    40    high_infl = cd > 4 / n
    41    ret[["removed.n"]]=sum(high_infl) 
    42    ret[["removed.fraction"]]=mean(high_infl)
    43    #Refit the multiple regression model without any influential points
    44    formula=as.formula(as.character(model$call[2]))
    45    model_new = lm(formula, data = data1, subset = !high_infl)
    46    ret[["new.model"]]=model_new
    47    par(mfrow=c(1,2))
    48    plot(fitted(model_new), resid(model_new), col = "dodgerblue", 
    49         xlab = "Fitted", ylab = "Residuals", main = "Fitted versus Residuals with Box-cox")
    50    abline(h = 0, col = "darkorange", lwd = 2)
    51    qqnorm(resid(model_new), main = "Normal Q-Q Plot with Box-cox", col = "dodgerblue")
    52    qqline(resid(model_new), col = "dodgerblue", lwd = 2)
    53    return(ret)
    54  }
    55  
    56  
    57  diagnostics = function(model, pcol="dodgerblue",lcol="orange",alpha=0.05,plotit=TRUE){
    58    
    59    if (plotit ){
    60      #fitted vs. residual
    61      par(mfrow=c(1,2))
    62      plot( model$fitted.values,
    63            model$residuals, 
    64            col=pcol,
    65            xlab="Fitted",
    66            ylab="Residuals",
    67            main="Fitted versus residuals" )
    68      abline(h=0,col=lcol,lwd=1)
    69      
    70      #QQ plot
    71      qqnorm(resid(model),main="Normal Q-Q Plot",col=pcol)
    72      qqline(resid(model),col=lcol,lwd=1)
    73    
    74    }
    75  
    76  }
    77  
\end{verbatim}


\end{document}
